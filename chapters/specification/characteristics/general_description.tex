\subsection{Общее описание}

Объектом автоматизации является игровой процесс с использованием ролевой системы Dungeons \& Dragons 3.5 редакции, а также игровой процесс, проводимый с использованием данной системы.

Суть любой ролевой системы в математическом обеспечении процесса игры. Наиболее часто такие системы применяются для проверки успешности действия или определения результата этого действия, однако этим использование ролевой системы в игровом процессе не ограничено.

В качестве базовой функции ролевой системы можно обозначить описание отдельных аспектов игры (таких, например, как характеристики персонажа), которые используются как при расчете каких-либо действий, так и для описания (например, <<крепкий стол>> может иметь прочность равную десяти, в то время как <<хлипкий стол>> может иметь прочность, равную 3).

Для определения параметров в процессе игры (таких, например, как успешность некоторого действия) используются генераторы случайных чисел. В качестве таких генераторов используется набор игральных костей.

Игровой процесс, проводимый в соответствии с правилами ролевой системы Dungeons \& Dragons, включает в себя несколько стадий и определяет роли участников в данном процессе.

Для каждой игры определены две основные роли: игрок и мастер. Игрок~--- тот, кто участвует в процессе игры через управление персонажем. Первая задача, стоящая перед игроком~--- создание персонажа. Для этого игрок определяет характер и основные параметры персонажа, которые потом будут использоваться в игре. Основная задача игрока, следующая за созданием персонажа~--- непосредственно игра, то есть управление персонажем. Для этого в процессе игры игрок обозначает действия, которые пытается совершить его персонаж, после чего в зависимости от типа действия определяет успешность этого действия с помощью бросков игральных костей.

В игре может участвовать от одного до нескольких игроков. Число игроков является произвольным и не регламентировано правилами. Группа игроков, участвующих в одной игре называется партией.

Мастер игры определяет сценарий и основные параметры игры, в т.ч. игровой сеттинг, время и место событий. Для каждой игры нужен только один мастер. В задачи мастера входит управление процессом игры, координирование действий игроков. Также в задачи мастера входит определение адекватности действий игроков сеттингу и правилам, помощь в разрешении неясных или конфликтных ситуаций.

Каждая отдельно взятая игра характеризуется целью, которую в ходе этой игры необходимо выполнить. Достижение цели проходит в рамках игровой кампании. Длительность кампании правилами не регламентируется и, в зависимости от цели, может быть от одного-двух дней до нескольких лет.

Кампания состоит из игровых сессий. Игровая сессия — это отрезок времени, в пределах которого ведётся игра. В одной кампании может быть от одной до нескольких сессий. Каждая сессия в среднем длится от двух до десяти часов. После каждой игровой сессии мастер игры определяет предварительные результаты и фиксирует текущее состояние внутриигрового мира.
