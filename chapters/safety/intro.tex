\subsection*{Общие сведения}

Автоматизация и компьютеризация труда человека затронула почти все области его жизнедеятельности.
В настоящее время ни одна организация не может функционировать в полной мере без применения ЭВМ и специализированной компьютерной техники.
Это, в свою очередь, приводит к тому, что значительное число работников проводят полный рабочий день за ЭВМ.

Однако тенденции компьютеризации общества несут в себе не только явные выгоды, но и угрозу здоровью --- актуальной становится проблема охраны труда человека, сохранение его работоспособности и благополучия.

Охрана труда обеспечивается системой законодательных актов, социально-экономических, организационных, технических, гигиенических и лечебно-профилактических мероприятий и средств, направленных на создание таких условий труда, при которых исключено или максимально снижено воздействие на работающих опасных и вредных производственных факторов.

Создание наиболее благоприятных, комфортных условий труда, улучшение охраны труда и техники безопасности, без сомнения, ведет к более высокой производительности труда, социальному развитию и повышению благосостояния.

Разработка и создание принципиально новой безопасной, безвредной для человека технологии, современных коллективных и индивидуальных средств защиты от опасных и вредных производственных факторов --- основные направления деятельности в области охраны труда.

Основное содержание этого раздела посвящено безопасности и охране труда человека при эксплуатации персональной ЭВМ.
