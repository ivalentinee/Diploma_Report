\subsection{Охрана труда}
\subsubsection{Анализ вредных и опасных производственных факторов}
Вредный производственный фактор --- производственный фактор, воздействие которого на работника может привести к его заболеванию.

Опасный производственный фактор --- фактор, воздействие которого на работника может привести к его травме.

\portraitImg[h!]{0.9}
            {images/safety}
            {Принципиальная блок-схема обеспечения безопасности объекта проектирования}
            {safety}

На рис. \ref{img:safety} приведена принципиальная блок-схема обеспечения безопасности объекта проектирования.
Основные вредные факторы, влияющие на состояние здоровья людей, работающих за компьютером:
\begin{itemize}
  \item{воздействие электромагнитного излучения монитора;}
  \item{утомление глаз, нагрузка на зрение;}
  \item{перегрузка суставов и мышц;}
  \item{стресс при потере информации.}
\end{itemize}
У людей, работающих за компьютером, наибольшее число жалоб на здоровье связано с заболеваниями мышц и суставов.
Чаще всего это онемение шеи, боль в плечах и пояснице или покалывание в ногах. Но бывают, однако, и более серьезные заболевания. Наиболее распространен кистевой туннельный синдром, при котором нервы руки повреждаются вследствие частой и длительной работы на компьютере. Это может привести к повреждению суставного и связочного аппарата кисти, а в дальнейшем заболевания кисти могут стать хроническими.
