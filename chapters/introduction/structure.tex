В первом разделе пояснительной записки к дипломному проекту приводится техническое задание на разрабатываемую систему. Здесь рассматривается процесс проектирования программного обеспечения для системы автоматизированного управления ПО, формулируются основные требования к функциям разрабатываемой системы и к видам ее обеспечения, анализируется актуальность проводимой разработки.

Раздел \ref{sec:existing_system} посвящен информационному моделированию разрабатываемой системы и содержит описание ряда процессов, протекающих в ней.

В разделе~\ref{sec:dataware} выполнен обоснованный выбор СУБД. Также в этом разделе преведены диаграммы структуры базы данных логического и физического уровня, описаны сущности базы данных.

Раздел~\ref{sec:math} содержит описание наиболее сложных и нетиповых алгоритмов, используемых информационной системой.

В разделе \ref{sec:software} описываются структура программного обеспечения, функции его компонентов, обосновывается выбор компонентов ПО, приводится описание модулей системы. Также в данном разделе описано руководство пользователя.

Раздел \ref{sec:hardware} посвящен выбору аппаратного обеспечения для работы системы, а также для её разработки.

Тестированию системы посвящён раздел \ref{sec:testing}. В нём описаны способы и особенности тестирования, описание исходных данных для тестирования и покрытие кода.

Разделы \ref{sec:economics} и \ref{sec:safety} содержат в себе аналитическую оценку экономической целесообразности проекта и оценку безопасности и экологичности проекта.
