\subsubsection{Логическая модель данных}

\begin{longtable}[h]{| p{0.04\textwidth} | p{0.3\textwidth} | p{0.5\textwidth} |}
\caption{\label{tab:logic_db_struct}Сущности логической модели данных.} \\
  \hline
  №  &  Название сущности    &  Описание       \\
\endfirsthead
\tableContinue{3}
  \\ \hline
\endhead
  \hline
  1  &  Пользователь          &   Данные пользователя                                               \\
  \hline
  2  &  Данные аутентификации &   Информация, необходимая для аутентификации пользователя в системе \\
  \hline
  3  &  Игра                  &   Данные игры                                                       \\
  \hline
  4  &  Сессия                &   Данные игровой сессии                                             \\
  \hline
  5  &  Приглашение в игру    &   Приглашение в игру                                                \\
  \hline
  6  &  Комментарий           &   Предмет и содержании комментария                                  \\
  \hline
  7  &  Персонаж              &   Общие данные персонажа                                            \\
  \hline
  8  &  Уровень персонажа     &   Данные персонажа, характерные для конкретного уровня              \\
  \hline
  9  &  Раса                  &   Данные игровой расы                                               \\
  \hline
  10 &  Класс                 &   Данные игрового класса                                            \\
  \hline
  11 &  Классовый уровень     &   Данные уровня игрового класса                                     \\
  \hline
  12 &  Свойство              &   Параметр сущности предметной области                              \\
  \hline
  13 &  Модификатор           &   Значение свойства объекта                                         \\
  \hline
\end{longtable}

\subimport{logical_model/}{user}
\subimport{logical_model/}{auth_data}
\subimport{logical_model/}{game}
\subimport{logical_model/}{session}
\subimport{logical_model/}{invitation}
\subimport{logical_model/}{comment}
\subimport{logical_model/}{character}
\subimport{logical_model/}{character_level}
\subimport{logical_model/}{race}
\subimport{logical_model/}{class}
\subimport{logical_model/}{class_level}
\subimport{logical_model/}{property}
\subimport{logical_model/}{modifier}
