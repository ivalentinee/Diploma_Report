\subsubsection{Модуль \textbf{Аутентификация пользователей в системе}}

Данный модуль предоставляет возможность аутентификации в системе. Для аутентификации используется сервис Google. Модуль содержит 30 строк кода.

\begin{longtable}[h]{| p{0.42\textwidth} | p{0.08\textwidth} | p{0.42\textwidth} |}
\caption{\label{tab:auth_data_files}Файлы модуля \textbf{Аутентификация пользователей в системе}.} \\
  \hline
  \textbf{Имя файла}  &  \textbf{Число строк}  &  \textbf{Описание} \\
\endfirsthead
\tableContinue{3} \\
  \hline
  \textbf{Имя файла}  &  \textbf{Число строк}  &  \textbf{Описание} \\
  \hline
\endhead
  \hline
  app/controllers/web/games/ sessions\_controller.rb  &  28  &  Контроллер, отвечающий за обработку аутентификационных данных \\
  \hline
  app/models/auth\_data.rb  &  2  &  Модель, отвечающая за хранение аутентификационных данных \\
  \hline
\end{longtable}

\begin{longtable}[h]{| p{0.04\textwidth} | p{0.41\textwidth} | p{0.46\textwidth} |}
\caption{\label{tab:auth_data_specification}Спецификация модуля \textbf{Аутентификация пользователей в системе}.} \\
  \hline
  \textbf{№}  &  \textbf{Название и тип элемента}  &  \textbf{Описание} \\
\endfirsthead
\tableContinue{3} \\
  \hline
  \textbf{№}  &  \textbf{Название и тип элемента}  &  \textbf{Описание} \\
  \hline
\endhead
  \hline
  \multicolumn{3}{|c|}{\textbf{Подпрограммы}} \\
  \hline
  1  &  create   &  Аутентификация пользователя \\
  \hline
  2  &  destroy  &  Деаутентификация пользователя \\
  \hline
\end{longtable}
