\subsection{Выбор конфигурации и параметров компьютера}

\subsubsection{Выбор конфигурации ПЭВМ для разработки}

Для разработки системы необходимо выбрать недорогую портативную ПЭВМ с достаточными для разработки по мощности комплектующими.

Так как разработка ведётся с использованием ruby и RubyOnRails в текстовом редакторе Emacs, требования к компрелктующим невысоки. Однако использование виртуалицации требует процессора с соответствующей технологией (например, Intel VT-x) и достаточно большого объёма памяти.

Наиболее подходящие комплектующе для разработки указаны в таблице~\ref{tab:pc_configuration}.

\begin{longtable}[h]{| p{0.3\textwidth} | p{0.6\textwidth} |}
\caption{\label{tab:pc_configuration} Комплектующие ПЭВМ для разработки } \\
  \hline
  \textbf{Тип комплектующего}  &  \textbf{Наименование} \\
\endfirsthead
\tableContinue{2} \\
  \hline
  \textbf{Тип комплектующего}  &  \textbf{Наименование} \\
  \hline
\endhead
  \hline
  Процессор           &  Intel Core i3-2350M    \\
  \hline
  Оперативная память  &  8 Гб DDR3 1600 МГц     \\
  \hline
  Видеокарта          &  Intel HD Graphics 3000 \\
  \hline
\end{longtable}

%% \subsubsection{Выбор конфигурации серверной ЭВМ}
