\subsection{Калькуляция себестоимости и расчет отпускной цены продукта}

\begin{longtable}[h]{| p{0.7\textwidth} | p{0.2\textwidth} |}
\caption{\label{tab:costing}Калькуляция себестоимости.} \\
  \hline
   Наименование элемента затрат      &  Затраты, руб       \\
\endfirsthead
\tableContinue{2}
  \\ \hline
\endhead
  \hline
   Сырьё и материалы                 & 115~240.00          \\
  \hline                                                
   Электроэнергия                    & 1~638.68            \\
  \hline                                                
   Оплата труда                      & 141~440.00          \\
  \hline                                                
   Амортизация                       & 3~434.15            \\
  \hline                                                
   Отчисления во внебюджетные фонды  & 42~714.88           \\
  \hline                                                
   Другие затраты                    & 24~540.49           \\
  \hline
  \textbf{ИТОГО}                     & \textbf{329~008.20} \\
  \hline
\end{longtable}

Плановая прибыль рассчитывается по формуле:
\formulka{ {{C_{\text{пол}} \cdot \text{Р}_{\text{н}}} \over {100}} }

Для данного проекта она составит:

\formulka{ {{329008.20 \cdot 20} \over {100}} = 65 801.64 \fUnitT{руб.} }

Таким образом, полную стоимость проекта можно определить как:

\formulka{ \text{С}_{\text{пр}} = \text{С}_{\text{пол}} + \text{П}}

\formulka{ \text{С}_{\text{пр}} = 329008.20 + 61 001.64 = \textbf{390 009.84} \fUnitT{руб.}}
