\subsubsection{Материальные затраты}

Затратами на сырьё и материалы являются затраты на специализированные устройства с дисплеями (три планшета, три смартфона) и расходные материалы, включающие в себя бумагу для печати, канцтовары и одну заправку картриджа лазерного ЧБ принтера, необходимую для печати. Материальные затраты на разработку показаны в таблице~\ref{tab:mat_expenditures}.

\begin{longtable}[h]{| p{0.2\textwidth} | p{0.2\textwidth} | p{0.1\textwidth} | p{0.15\textwidth} | p{0.1\textwidth} |}
\caption{\label{tab:mat_expenditures}Затраты на сырьё и материалы в процессе реализации и тестирования программного продукта.} \\
  \hline
  \textbf{Параметр}  &  \textbf{Модель}  &  \textbf{Цена, руб} &  \textbf{Количество единиц, шт.}  &  \textbf{Сумма, руб}  \\
\endfirsthead
\tableContinue{5} \\
  \hline
  \textbf{Параметр}  &  \textbf{Модель}  &  \textbf{Цена, руб} &  \textbf{Количество единиц, шт.}  &  \textbf{Сумма, руб}  \\
  \hline
\endhead
  \hline
   Серверная платформа OS Linux  &  ASUS RS704D-E6/P     &  90~092     &                       1  &  90~092      \\
  \hline
   Планшет с ОС Android          &  Google NEXUS 7 16gb  &  10~288     &                       1  &  10~288      \\
  \hline
   Ноутбук с ОС Linux            &  Asus X501A           &  11~560     &                       1  &  11~560      \\
  \hline
  \multicolumn{4}{|r|}{\textbf{ИТОГО}}                                                     & \textbf{111~940}    \\
  \hline
\end{longtable}


Основными материальными затратами при разработке систему информационного обеспечения и управления для ролевой игры \dnd~ являются затраты электроэнергии на питание персональных компьютеров, на которых производится реализация программного продукта. При мощности блока питания серверной платформы в 770 Вт затраты электроэнергии всех устройств серверной платформы будут составлять примерно \textbf{750 Вт/ч}. При использовании максимальной мощности ноутбука, затраты электроэнергии будет составлять примерно \textbf{67 Вт/ч}. При работе подключенных к сети электропитания остальных устройств (планшет) происходит расход энергии \textbf{28 Вт/ч}.

\textit{Затраты электроэнергии на ПК для реализации проекта:}\\
\formulka{ \text{Э}_{\text{ПК}} = (750 + 67 + 28) \cdot 8 \cdot 21.25 \cdot 4 = 574.60 \fUnitT{кВт} }

Тариф на электроэнергию: 1 кВт --- 2.51 руб

\textit{Стоимость затраченной на ПК электроэнергии для реализации проекта:}\\
\formulka{ \text{Ц}_{\text{Эпк}} = 574.60 \cdot 2.51 = 1~442.25 \fUnitT{руб.} }

На рабочем месте программиста должна быть обеспечена необходимая с учётом времени года освещенность (нормы естественного, искусственного и совмещённого освещения зданий и сооружений приведены в СНиП 23-05-95 <<Естественное и искусственное освещение>>) посредством общего и местного искусственного освещения. Расход электроэнергии на искусственное освещение помещения и рабочего места программиста на период разработки программного продукта с февраля по май включительно представлены в таблице~\ref{tab:energy_expenditures}.

\begin{landscape}
  \begin{longtable}{|p{0.17\textwidth}|p{0.17\textwidth}|p{0.17\textwidth}|p{0.17\textwidth}|p{0.17\textwidth}|p{0.17\textwidth}|p{0.17\textwidth}|}
  \caption{\label{tab:energy_expenditures}Расход электроэнергии на искусственное освещение рабочего места программиста.} \\
    \hline
     \textbf{Месяц}  &  \textbf{Суточная длительность местного освещения}  &  \textbf{Расход электроэнергии на местное освещение}  &  \textbf{Суточная длительность общего освещения}  &  \textbf{Расход электроэнергии на общее освещение}  &  \textbf{Рабочих дней}  &  \textbf{Расход электроэнергии на общее и местное освещение}  \\
    \hline
     февраль  &                4  &               75  &                6  &              150  &               20  &            24.00  \\
    \hline
     март     &              3.5  &               75  &                5  &              150  &               21  &            21.26  \\
    \hline
     апрель   &                3  &               75  &                4  &              150  &               22  &            18.15  \\
    \hline
     май      &                3  &               75  &                3  &              150  &               22  &            14.85  \\
    \hline
    \multicolumn{6}{|r|}{\textbf{ИТОГО}}                                                     & \textbf{78.26}     \\
    \hline
  \end{longtable}
\end{landscape}

\textit{Стоимость затраченной на освещение одного места электроэнергии для реализации проекта:}
\formulka{ \text{Ц}_{\text{Эосв}} = 78.26 \cdot 2.51 = 196.43 \fUnitT{руб.} }


\textbf{\textit{Общие затраты на электроэнергию:}}
\formulka{ \text{Ц}_{\text{ЭЛЕКТР.}} = 1~442.25 + 196.43 = \textbf{1~638.68} \fUnitT{руб.} }
