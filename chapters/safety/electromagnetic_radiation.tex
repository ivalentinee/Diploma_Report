\subsection{Электромагнитное излучение}
Электромагнитным полем называется особая форма материи, посредством которой осуществляется взаимодействие между электрически заряженными частицами.

Токоведущие части действующих установок являются источником электромагнитных полей промышленной частоты.
Длительное действие электромагнитного поля на организм вызывает нарушение функционального состояния центральной нервной системы, сердечно-сосудистой системы, что приводит к быстрому утомлению, уменьшению работоспособности, болям в области сердца, изменению кровяного давления.

Параметрами степени воздействия электромагнитных полей на человека являются:
\begin{itemize}
  \item{интенсивность излучения;}
  \item{режим излучения;}
  \item{длина волны;}
  \item{размер облучаемой поверхности тела;}
  \item{особенности облучаемого организма;}
  \item{продолжительность воздействия.}
\end{itemize}

Для пользователя ПЭВМ основным источником электромагнитных полей и ионизирующих излучений является электронно-лучевая трубка дисплея.

Электромагнитные поля оказывают тепловое воздействие на организм, приводящее к структурным и функциональным изменениям в нем.

Воздействие электромагнитных полей способно вызвать изменение клеток и состава крови, замутнение хрусталика глаза, выпадение волос, ломку ногтей, ожоги, кожные заболевания и др.

Временно допустимые уровни электромагнитного потока на рабочем месте не должны превышать (для диапазона частот 5 Гц --- 2 кГц):
\begin{itemize}
  \item{плотность магнитного потока --- не более 250 нТл (по данным СанПиН 2.2.2./2.4.1340-03);}
  \item{напряженность электрического поля --- не более 25 В/м;}
  \item{радиационное излучение на расстоянии 5 см от экрана --- не более 100 мкР/ч. (по данным СанПиН 2.2.2./2.4.1340-03).}
\end{itemize}

Электромагнитные поля радиочастот делятся на высокие частоты (ВЧ), ультравысокие частоты (УВЧ), сверхвысокие частоты (СВЧ).

На пользователя, работающего за монитором с ЭЛТ-дисплеем, воздействует практически весь диапазон излучений, оказывая биологическое и тепловое воздействие.

Во всех случаях предельно допустимое значение плотности потока энергии не должно превышать 10 мкВт/$\textup{см}^2$.

Однако, в настоящее время мониторы с ЭЛТ почти повсеместно вышли из употребления, будучи заменены на более безопасные мониторы с ЖК-дисплеем.
Эти мониторы характеризуются нулевым уровнем ионизирующих излучений, однако, тем не менее, создают повышенную нагрузку на ЦНС человека за счет нагрузки на зрительную систему.

\subsubsection{Электрический ток}

Действие электрического тока на органы человека может быть тепловым (ожог), механическим (разрыв тканей), химическим (электролиз) и биологическим (сокращение мышц, паралич дыхания и сердца).

При воздействии электрического тока на органы человека могут быть два вида поражения: электрические удары и электрические травмы.

Электрические удары --- это возбуждение живых тканей организма протекающим через него электрическим током, проявляющееся в непроизвольных судорожных сокращениях различных мышц тела.
Они разделяются условно на четыре группы:
\begin{enumerate}
  \item{судорожное сокращение мышц без потери сознания;}
  \item{судорожное сокращение мышц с потерей сознания;}
  \item{потеря сознания и нарушение сердечной деятельности или дыхания;}
  \item{клиническая смерть, т.е. отсутствие дыхания и кровообращения.}
\end{enumerate}

По степени опасности поражения людей током рабочее пространство, оборудование дисплеями, относится к классу помещений без повышенной опасности согласно ПУЭ.

Основное питание в помещении должно осуществляется от трехфазной цепи током с частотой 50 Гц и напряжением 220 В.
Сеть трёхфазная с заземленной нейтралью ГОСТ 12.1.030-81.

В соответствии с ГОСТ 12.1.030-81, в помещениях с электрооборудованием должна быть проложена шина защитного заземления (заземляющий проводник сечением не менее 120 мм), который должен металлически соединяться с заземляющей нейтралью электроустановок, от которой осуществляется электропитание оборудования.
Сопротивление заземляющего устройства, с которым соединяется нейтраль, должно быть не более 4 Ом.

Шина защитного заземления должна быть доступна для осмотра.

В соответствии с ГОСТ 12.1.030-81 подводка питания к дисплеям и устройствам должна осуществляться под съемным полом или в каналах.
Согласно ГОСТ 12.1.030-81, электрооборудование помещения относится к 1 классу защиты от поражения электрическим током, т.е. имеется рабочая изоляция, элемент для заземления и провод без зануляющей шины для подсоединения к источнику питания.

Для защиты персонала от попадания под опасное напряжение при неисправной изоляции необходимо предусмотреть защитное заземление, выполняемое в соответствии с ГОСТ 12.1.03-81.

Предупреждение коротких замыканий в электрической сети обеспечивается правильным выбором проводов (выбор сечения, токоведущих шин, марки проводов, видов изоляции); профилактические осмотры, ремонты.
Для быстрого отключения при коротком замыкании в комнате используются плавкие предохранители.

В случае аварийных ситуаций предусмотрено защитное отключение ПЭВМ за счет превращения тока замыкания на корпус в ток однофазного короткого замыкания с последующим срабатыванием защиты.

В рассматриваемом помещении установка всего электрооборудования выполнена в соответствии с нормами электробезопасности.
Питание осуществляется от трехфазной цепи с заземленной нетралью током с частотой 50 Гц и напряжением 220 В.

Электроснабжение всех потребителей осуществляется через главный распределительный щит.
Разводка электропитания по потребителям энергии осуществляется с помощью установленных в помещениях распределительных щитов и розеток.

Для отключения питания электрической сети в помещениях предусматриваются рубильники.

То есть, такой фактор как электробезопасность рассмотренной работы, а именно работа с персональным компьютером согласно ГОСТ 12.1.030-81 можно отнести по степени опасности к допустимым условиям труда.
В общем электробезопасность удовлетворяет ГОСТ 12.1.030-81 и дополнительных мер по защите не требуется.

\subsubsection{Требования по обеспечению пожарной безопасности}

Противопожарная защита имеет целью поиска самых экономически целесообразных и технически обоснованных способов и средств предупреждения пожаров и их ликвидации с минимальным ущербом при оптимальном использовании сил и технических средств тушения.

Пожарная безопасность --- это состояние объекта, при котором исключается возможность пожара, а в случае его возникновения используются необходимые меры по устранению негативного влияния опасных факторов пожара на людей, сооружения и материальных ценностей.

Пожарная безопасность может быть обеспечена мерами пожарной профилактики и активной пожарной защиты.
Пожарная профилактика включает комплекс мероприятий, направленных на предупреждение пожара или уменьшение его последствий.

Активная пожарная защита --- меры, обеспечивающие успешную борьбу с пожарами или взрывоопасной ситуацией.

В помещениях запрещается:
\begin{itemize}
  \item зажигать огонь;
  \item включать электрооборудование, если в помещении пахнет газом;
  \item курить;
  \item сушить что-либо на отопительных приборах;
  \item закрывать вентиляционные отверстия в электроаппаратуре.
\end{itemize}

Источниками воспламенения являются:
\begin{itemize}
  \item искра при разряде статического электричества
  \item искры от электрооборудования
  \item искры от удара и трения
  \item открытое пламя
\end{itemize}

При возникновении пожароопасной ситуации или пожара персонал должен немедленно принять необходимые меры для его ликвидации, одновременно оповестить о пожаре администрацию.

По взрывопожарной и пожарной опасности помещения делятся по категориям А, Б, В, Г, Д (в зависимости от свойств применяемых веществ и материалов). Исследуемое  помещение относится к категории В (помещение содержит горючие и трудногорючие жидкости, твердые горючие и трудногорючие вещества в малом количестве и материалы, способные только гореть при взаимодействии с кислородом воздуха).

В зависимости от пределов огнестойкости строительных конструкций установлены восемь степеней огнестойкости зданий: I, II, III, IIIа, IIIб, IV, IVа, V. Учитывая высокую стоимость оборудования, а также категорию пожароопасности, здание, в котором предусмотрено размещение ПЭВМ, должно быть отнесено к I или II степени огнестойкости согласно ГОСТ 12.1.004-89. По классификации класс пожароопасных зон относится к II-2-А (зона в которой обращаются твердые горючие вещества) согласно ПУЭ.

К горючим материалам, присутствующим в исследуемом помещении относятся: строительные материалы для акустической и эстетической отделки, двери, полы, изоляции силовых, сигнальных кабелей, обмотки радиотехнических деталей, изоляции соединительных кабелей, ячеек, блоков, панелей, жидкости для очистки элементов, узлов ПЭВМ от загрязнения и другие.

В качестве средств пожаротушения предусмотрены: огнетушитель и система автоматической пожарной сигнализации.

Работа с персональным компьютером удовлетворяет ГОСТ 12.1.004-89 в вопросах пожароопасности и взрывоопасности, следовательно, специализированнх дополнительных мер по защите не требуется.

\subsubsection{Психофизиологический фактор}

Психофизиологическое обеспечение профессиональной деятельности является важным звеном в комплексе мероприятий, проводимых инспекторами по охране труда в интересах сохранения здоровья и профессионально надежной деятельности специалистов, повышая безопасность и эффективность трудовой деятельности. 

К психофизиологическим факторам условий труда относятся: 

\begin{itemize}
  \item напряжение зрения и внимания;
  \item интеллектуальные и эмоциональные нагрузки;
  \item длительные статические нагрузки и монотонность труда.
\end{itemize}

Зрительная система человека приспособлена для восприятия картин природы, рисунков и печатных текстов, а не для работы с дисплеем. Изображение на дисплее состоит из светящихся и мерцающих дискретных точек, что принципиально отличается от привычных глазу объектов наблюдения.

При работе на компьютере часами у глаз не бывает необходимых фаз расслабления, глазные мышцы напрягаются, их работоспособность снижается. Большую нагрузку орган зрения испытывает при вводе информации, так как пользователь вынужден часто переводить взгляд с экрана на текст и клавиатуру, находящиеся на разном расстоянии и по-разному освещенные. 

Зрительное утомление выражается в:

\begin{itemize}
  \item двоении;
  \item затуманивании зрения;
  \item кажущемся изменении окраски предметов;
  \item трудности при переносе взгляда с ближних на дальние и с дальних на ближние предметы;
  \item неприятных ощущениях в области глаз (чувства жжения, <<песка>>), покраснении век, боли при движении глаз.
\end{itemize}

Системный администратор зачастую занят монотонным трудом при набирании больших количеств кода. К сожалению, даже при всех современных методиках разработки программного обеспечения встречаются участки работы, где без этого не обойтись. 

Для уменьшения утомления, связанного с влиянием этого фактора необходимо применять оптимальные режимы труда и отдыха в течение рабочего дня. Одна из распространенных рекомендаций --- при работе за компьютером необходимо делать 15-минутные перерывы через каждые два часа, а при интенсивной работе --- через каждый час. Также рекомендуется периодически выполнять несколько физических упражнений по своему вкусу.

Рабочая поза оказывает значительное влияние на эффективность работы человека. Рабочее место --- зона, оснащенная необходимыми техническими средствами, в которой совершается трудовая деятельность исполнителя или группы исполнителей, совместно выполняющих одну работу.

Под техническими средствами понимается основное и вспомогательное оборудование, устройства техники безопасности, санитарно гигиенические, культурно-бытовые устройства, необходимые для наиболее экономичного или наиболее производственного выполнения определенных технологических операций.

Под организацией рабочего места понимается система мероприятий по оснащения рабочего места средствами и предметами труда и их размещению в определенном порядке.

Как правило, оператор ПЭВМ не имеет возможности выбирать помещение или делать кардинальные перестановки оборудования для обеспечения наиболее оптимальных условий работы.
Поэтому рекомендации данной главы следует рассматривать как исходный материал или руководство к действию для организаторов работ с использованием ПЭВМ.
Они могут быть использованы операторами, заинтересованными в усовершенствовании своих рабочих мест.
Все приведенные требования включены в нормативный документ Госкомсанэпиднадзора <<Гигиенические требования к ПЭВМ и организации работы. Санитарные правила и нормы>>, вступивший в силу в июне 2003 г. (СанПин 2.2.2/2.4.1340-03).

В производственных помещениях, в которых работа с использованием ПЭВМ является вспомогательной, температура, относительная влажность и скорость движения воздуха на рабочих местах должны соответствовать действующим санитарным нормам микроклимата производственных помещений. Рекомендуемая температура воздуха составляет 21 $С^o$, при относительной влажности 55\%, абсолютной влажности --- 10 ${\textup{г} \over \textup{м}^3}$ и скорости движения воздуха менее 0.1 ${\textup{м} \over \textup{с}}$

Планировка рабочего места должна удовлетворять требованиям удобства работы и экономии энергии и времени оператора, соблюдения правил личной и производственной безопасности.
\begin{itemize}
  \item При размещении рабочих мест с ПЭВМ расстояние между рабочими столами с видеомониторами (в направлении тыла поверхности одного видеомонитора и экрана другого видеомонитора), должно быть не менее 2,0 м, а расстояние между боковыми поверхностями видеомониторов --- не менее 1,2м.
  \item Рабочие места с ПЭВМ при выполнении творческой работы, требующей значительного умственного напряжения или высокой концентрации внимания, рекомендуется изолировать друг от друга перегородками высотой 1,5-2,0 м.
  \item Рабочие места с ПЭВМ в помещениях с источниками вредных производственных факторов должны размещаться в изолированных кабинах с организованным воздухообменом.
  \item Конструкция рабочего стола должна обеспечивать оптимальное размещение на рабочей поверхности используемого оборудования с учетом его количества и конструктивных особенностей, характера выполняемой работы. При этом допускается использование рабочих столов различных конструкций, отвечающих современным требованиям эргономики.
  \item Конструкция рабочего стула (кресла) должна обеспечивать поддержание рациональной рабочей позы при работе на ПЭВМ, позволять изменять позу с целью снижения статического напряжения мышц шейно-плечевой области и спины для предупреждения развития утомления. Тип рабочего стула (кресла) следует выбирать с учетом роста пользователя, характера и продолжительности работы с ПЭВМ.
  \item Рабочий стул (кресло) должен быть подъемно-поворотным, регулируемым по высоте и углам наклона сиденья и спинки, а также расстоянию спинки от переднего края сиденья, при этом регулировка каждого параметра должна быть независимой, легко осуществляемой и иметь надежную фиксацию.
  \item Поверхность сиденья, спинки и других элементов стула (кресла) должна быть полумягкой, с нескользящим, слабо электризующимся и воздухопроницаемым покрытием, обеспечивающим легкую очистку от загрязнений.
  \item Экран видеомонитора должен находиться от глаз пользователя на расстоянии 600 -- 700 мм, но не ближе 500 мм с учетом размеров алфавитно-цифровых знаков и символов.
\end{itemize}

\subsubsection{Требования к персоналу, эксплуатирующему средства вычислительной техники}

К самостоятельной эксплуатации электроаппаратуры допускается только специально обученный персонал не моложе 18 лет, пригодный по состоянию здоровья и квалификации к выполнению указанных работ.

Перед допуском к работе персонал должен пройти вводный и первичный инструктаж по технике безопасности с показом безопасных и рациональных примеров работы. Затем не реже одного раза в 6 месяцев проводится повторный инструктаж, возможно, с группой сотрудников одинаковой профессии в составе не более 20 человек. Внеплановый инструктаж проводится при изменении правил по охране труда, при обнаружении нарушений персоналом инструкции по технике безопасности, изменении характера работы персонала.

В помещениях, в которых постоянно эксплуатируется электрооборудование должны быть вывешены в доступном для персонала месте инструкции по технике безопасности, в которых также должны быть определены действия персонала в случае возникновения аварий, пожаров, электротравм.

Руководители структурных подразделений несут ответственность за организацию правильной и безопасной эксплуатации средств вычислительной техники и периферийного оборудования, эффективность их использования; осуществляют контроль за выполнением персоналом требований настоящей инструкции по технике безопасности.

\subsubsection{Расчет освещенности}

Профессиональным заболеванием операторов и программистов является ухудшение зрения.
Так как параметры помещения, в котором ведется работа, и используемой техники (персональной ПЭВМ) удовлетворяют санитарным нормам, то особое внимание следует уделить освещенности рабочего места.

Расчет освещенности проводится с путем расчета коэффициента использования с использованием метода светового потока).
Он позволяет учесть прямую и отраженную составляющую светового потока от потолка, стен и рабочих поверхностей.

Имеются следующие исходные данные:
\begin{itemize}
  \item площадь помещения --- 2,5x4 м;
  \item высота подвеса светильников $h_\textup{св}$= 2,5 м;
  \item источник освещения --- лампа люминесцентная (ЛБА), яркость фона --- светлая;
  \item яркость объекта --- средняя;
  \item система освещения --- общая;
  \item коэффициент отражения побеленного потолка: $p_\textup{п}$= 0,7;
  \item коэффициент отражения стен, обклеенных обоями: $p_c$ = 0,5;
  \item коэффициент отражения расчетной поверхности: $p_p$ = 0,3.
\end{itemize}

По таблице <<Нормы освещенности при искусственном освещении и коэффициент естественного освещения (для 3 пояса светового климата РФ) при естественном и совмещенном освещении>> (СНиП 23-05-95), исходя из характеристик зрительной работы определяем разряд и подразряд зрительной работы как IV-В.
Данному разряду соответствует норма искусственного освещения при системе комбинированного освещения 300 лк.

Норма рабочего искусственного освещения составляет $E_\textup{он}$ = 400 лк. Коэффициент запаса равен $K_3$ = 1,5, высота подвеса светильников $h_\textup{св}$ = 2,2 м.

Определяем индекс помещения:

$i = {{a \cdot b} \over {(a + b) \cdot h_\textup{св}}} = {{2,5 \cdot 4} \over {(2,5 + 4) \cdot 2.5}} = {\approx 0,6}$

Тип лампы --- ЛБА, люминесцентная белого света, амальгамная.
Интерполированием находим коэффициент использования: $\eta$ = 0,52.

Определяем расстояние между светильниками и по нему --- число светильников в помещении.
Рекомендованное отношение $\lambda = I_\textup{св} \over h_\textup{св}$ равно 0,8--1,2. Принимаем $\lambda$ = 0,8, тогда  $I_\textup{св} = 0,8 \cdot 2,5 = 2$ м.

Число светильников при размещении по углам квадрата вычисляется по формуле:

$n = {{a \cdot b} \over {I_\textup{св}^2}} = {{2,5 \cdot 4} \over {4}} \approx 3$

Определяем световой поток одной лампы.
Коэффициент минимальной освещенности, зависящий от размещения и светораспределения светильников, создающих некоторую неравномерность распределения светового потока по расчетной плоскости, принимаем равным $Z$ = 1,1.

$\Phi_o = {{E \cdot S \cdot Z \cdot K_3} \over {n \cdot \eta}} = {{400 \cdot 10 \cdot 1,1 \cdot 1,5} \over {3 \cdot 0,52}} \approx 4231 (\textup{лм})$

Выбираем лампу ЛБ-80-7 со световым потоком 5200 лм, срок продолжительности горения 12000 час, мощность 80 Вт.
Суммарная мощность осветительной установки общего освещения:

$P = P_o \cdot n = 80 \cdot 3 = 240 (\textup{Вт})$

Таким образом, на площадь 2,5 x 4 м для работы за дисплеем при общем освещении должны использоваться 3 светильника по 1 лампе ЛБА-80-7.
