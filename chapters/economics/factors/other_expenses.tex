\subsubsection{Прочие расходы}

%% Отчисления во внебюджетные фонды

Отчисления на социальные нужды включают в себя следующие отчисления:
\begin{enumerate}
  \item Пенсионный фонд Российской Федерации (ПФР). На его долю приходится:
    \begin{enumerate}
      \item Взнос на страховую часть пенсии – 16 \%.
      \item Взнос на накопительную часть пенсии – 6 \%.
    \end{enumerate}
  \item Федеральный фонд обязательного медицинского страхования (ФФОМС). На его долю приходится 5.1 \%.
  \item Фонд социального страхования (ФСС). Взнос на обязательное страхование на случай временной нетрудоспособности и в связи с материнством. На его долю приходится 2.9 \%.
  \item Фонд социального страхования (ФСС). Взнос по страхованию от несчастных случаев на производстве и профзаболеваний. На его долю приходится 0.2 \% (минимальный тариф, максимальный --- 8.5 \%).
\end{enumerate}

В сумме данные отчисления составляют 30.2\% к фонду оплаты труда.
\formulka{ \text{Ц}_{\text{СОЦ ОТЧ}} = \rfrac{\text{ЦЗ}}{\text{П}} \cdot 0.302 = 141~440.00 \cdot 0.302 = 42~714.88 \fUnitT{руб.} }


%% Другие расходы
Расчёт затрат на водоснабжение ведётся по установленным тарифам с учётом месячного расхода (в случае с водоснабжением) или по среднему значению, учтённому в самом тарифе (в случае с канализацией).

\begin{longtable}[h]{| p{0.2\textwidth} | p{0.2\textwidth} | p{0.2\textwidth} | p{0.1\textwidth} | p{0.15\textwidth} |}
\caption{\label{tab:WS_sewer}Затраты на водоснабжение и канализацию.} \\
  \hline
   \textbf{Наименование}  &  \textbf{Затраты на одного рабочего}  &  \textbf{Длительность проекта}  &  \textbf{Тариф}  &  \textbf{Затраты}    \\
\endfirsthead
\tableContinue{5} \\
  \hline
  \textbf{Наименование}  &  \textbf{Затраты на одного рабочего}  &  \textbf{Длительность проекта}  &  \textbf{Тариф}  &  \textbf{Затраты}    \\
  \hline
\endhead
  \hline
   Водоснабжение   &  0.2 $\text{м}^3$/мес          &  4 мес.               &  16,87 руб/$\text{м}^3$ &  13.49 руб. \\
  \hline
   Канализаця      &  плата по тарифу               &  4 мес.               &  2.15 руб./мес.         &  8.60 руб.  \\
  \hline
   \multicolumn{4}{|r|}{\textbf{ИТОГО}}                                                     & \textbf{22.09 руб.}    \\
  \hline
\end{longtable}


Расчёт затрат на теплоснабжение и услуги ЖКХ ведётся, исходя из стоимости отопления 1\sqMeter помещения, рассчитанной на 1 рабочего (10\sqMeter --- площадь рабочего места программиста и 5\sqMeter --- общая площадь здания на одного человека). Расчет затрат на аренду помещения ведется по тарифу 400 руб./$\text{м}^2$/мес., за помещение 15\sqMeter.


\begin{longtable}[h]{| p{0.2\textwidth} | p{0.17\textwidth} | p{0.17\textwidth} | p{0.17\textwidth} | p{0.14\textwidth} |}
\caption{\label{tab:housing}Затраты на теплоснабжение и услуги ЖКХ.} \\
  \hline
  \textbf{Наименование}  &  \textbf{Площадь на одного рабочего}  &  \textbf{Длительность проекта}  &  \textbf{Тариф}  &  \textbf{Затраты}    \\
\endfirsthead
\tableContinue{5} \\
  \hline
  \textbf{Наименование}  &  \textbf{Площадь на одного рабочего}  &  \textbf{Длительность проекта}  &  \textbf{Тариф}  &  \textbf{Затраты}    \\
  \hline
\endhead
  \hline
   Теплоснабжение   &  15\sqMeter                &  4 мес.               &  1.75 $\rfrac{\text{руб.}}{\text{м}^2 \cdot \text{мес}.}$ &  105.00 руб.  \\
  \hline
   Услуги ЖКХ       &  15\sqMeter                &  4 мес.               &  6.89 $\rfrac{\text{руб.}}{\text{м}^2 \cdot \text{мес}.}$ &  413.40 руб.  \\
  \hline
   Аренда помещения &  15\sqMeter                &  4 мес.               &  400  $\rfrac{\text{руб.}}{\text{м}^2 \cdot \text{мес}.}$ &  24 000 руб.  \\
  \hline
   \multicolumn{4}{|r|}{\textbf{ИТОГО}}                                                                 & \textbf{24~518.40}    \\
  \hline

\end{longtable}


\textbf{\textit{Общие затраты на другие расходы:}}
\formulka{ \text{Ц}_{\text{ДРУГИЕ}} = 22.09 + 24~518.40 = 24~540.49 \fUnitT{руб.} }
