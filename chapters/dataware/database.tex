\subsection{Выбор средств управления данными}

%% mysql vs postgres vs ещё что-нибудь vs oracle

В рамках проектирования системы для выбора СУБД были выбраны основные критерии:
\begin{itemize}
\item СУБД должна быть реляционной;
\item должна работать на ОС Linux;
\item должна оспользовать язык SQL для совершения операций;
\item должна поддерживать все основные типы данных.
\end{itemize}

На основе указанных критериев были выбраны три СУБД:
\begin{enumerate}
\item PostgreSQL.
\item MySQL.
\item Oracle DataBase.
\end{enumerate}

Для дальнейшего выбора был проведён анализ по дополнительным критериям.

\begin{longtable}[h]{| m{0.35\textwidth} | m{0.15\textwidth} | m{0.15\textwidth} | m{0.15\textwidth} |}
\caption{\label{tab:db_compare}Сравнение СУБД.} \\
  \hline
  \textbf{Критерий}  &  \textbf{PostgreSQL}  &  \textbf{MySQL}  &  \textbf{Oracle Database} \\
\endfirsthead
\tableContinue{4} \\
  \hline
  \textbf{Критерий}  &  \textbf{PostgreSQL}  &  \textbf{MySQL}  &  \textbf{Oracle Database} \\
  \hline
\endhead
  \hline
  Платность  &  Бесплатная  &  Бесплатная  &  Платная  \\
  \hline
  Поддержка массивов  &  Да  &  Нет  &  Да  \\
  \hline
  Подробная документация  &  Да  &  Да  &  Да  \\
  \hline
  Наличие средства редактирования с современным пользовательским интерфейсом  &  Да  &  Да  &  Да  \\
  \hline
\end{longtable}
