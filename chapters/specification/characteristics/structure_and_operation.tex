\subsection{Структура и принципы функционирования}

Игра с использованием ролевой системы \dnd включает в себя следующие процессы:
\begin{enumerate}
\item Подготовка к игре\\
Ролевая система \dnd является достаточно сложной и гибкой как с алгоритмической так и с концептуальной точки зрения. По этой причине процесс подготовки к игре является важным этапом и включает в себя несколько подпроцессов:
\begin{enumerate}
\item Создание сценария игры\\
Этот подпроцесс осуществляется мастером игры. В ходе данного подпроцесса создаётся идея игры, описание, создаются локации, генерируются персонажы. Немаловажным является определение начальных параметров, которые определяют сложность игры.
\item Создание персонажа\\
Данный подпроцесс осуществляется каждым из игроков. В ходе данного процесса игрок, на основе выданных мастером данных, создаёт концепцию персонажа, на основе которой затем подбирает параметры в соответствии с правилами игры и сеттингом.
\end{enumerate}
\item Игра\\
Игра состоит из нескольких сессий. В ходе сессии игроки совершают внутриигровые действия, общаются, производят локальные расчёты. Каждое действие в игре совершается в соответствиями с правилами игры, однако, если какое-то действие в правилах не описано, его результат определяется мастером. В некоторых случаях каждое совершаемое действие записывается в протокол сессии.

В ходе первой сессии следует выделить особый подпроцесс: расчёты. Так как многие расчёты следует провести под руководством мастера, эти расчёты невозможно включить в этап <<подготовка к игре>>.
\item Определение результатов игры\\
Подведение итогов игры является важным этапом как для мастера, так и для игроков.

В процессе определения результатов игры делаются выводы о качестве игрового процесса, обращается внимание на совершённые ошибки и недочёты для дальнейшего улучшения игрового процесса. Также в ходе подведения итогов определяется непосредственное завершение сценария, мастером игры частично описывается дальнейшая <<судьба>> персонажей, определяется, чего смогли достичь игроки за данную игру.

Немаловажным подпроцессом является составление отчёта. Данный отчёт должен содержать в себе как выводы относительно игрового процесса (качество игры, описание основных проблем), так и описание внутриигровых достижений. Данный отчёт может быть использован для восстановления состояния другой игры в том случае, если она будет основана на том же сценарии (например, будет являться продолжением).
\end{enumerate}

%%%%%%%%%%%%%%%%%%%%%%%%%%%%%%%%%%%%%%%%

В игровом процессе выполняются следующие операции по сбору и обработке информации:
\begin{enumerate}
\item Обмен персональными данными\\
Для участия в игре и возможности взаимодействия мастер и игроки обмениваются именами, номерами телефонов и адресами электронной почт
ы.
\item Начальные параметры\\
При создании материалов для игры мастер должен выбрать начальные параметры, такие как сеттинг, стартовый уровень, ограничения на параметры персонажа и пр. Все данные, полученные на этом этапе, передаются игрокам для создания персонажей.

\item Создание игрового окружения\\
На основании начальных параметров мастер создаёт описание локаций, ключевых событий и НИП. Часть этой информации может быть выдана игрокам для создания более подходящих персонажей.
\item Создание персонажей\\
После того, как выбраны начальные параметры, они передаются игрокам для создания персонажей. При создании персонажей игроки, используя полученную от мастера информацию, создают концепцию и описание персонажа. Затем на основе этих данных а также игровых руководств выбираются раса и класс. После этого используются игральные кости для генерации параметров и проводятся расчёты. Вся информация заносится в лист персонажа.
\item Проверка листов персонажей\\
После того, как расчёты проведены, игроки передают мастеру листы персонажа (или их копии) для проверки. В том случае, если в листах персонажа содержатся ошибки, листы передаются игрокам на доработку.
\item Обмен информацией во время игры\\
На основе сгенерированной информации начинается игровой процесс. На этом этапе происходит обмен внутриигровой информацией между мастером и игроками. В ходе игры могут изменяться данные, указанные в листах персонажей. Также могут записываться протоколы игр, составляться отчёты о проведённых сессиях. В отчёты и протоколы могут входить как общее описание, так и полные записи действий игроков и мастера.
\item Завершение игры\\
При завершении игры может создаваться общеигровой отчёт. Все отчёты и протоколы передаются ответственному лицу (в качестве которого чаще всего выступает мастер) на архивацию.
\end{enumerate}
