\section*{Заключение}
\addcontentsline{toc}{section}{Заключение}

В данном дипломном проекте проведен анализ предметной области, была проведена оценка актуальности проекта, изучены аналоги, поставлены цели и задачи проекта.

В рамках дипломного проектирования были выполнены следующие задачи:
\begin{itemize}
\item разработаны алгоритмы расчёта;
\item создана база игровых материалов с возможностью добавления, просмотра, редактирования и удаления записей;
\item создано средство управления игровым процессом, котрое позволяет управлять играми и игровыми сессиями, изменять состав участников игр;
\item создан редактор персонажей;
\item обеспечен пользовательский доступ к системе с помощью алгоритма аутентификации, позаоляющего использовать сторонние сервисы.
\end{itemize}

Для выполнения задач была изучена предметная область, составлено описание существующей информационной системы.

Были изучены и применены современные технологии, такие как ruby, RubyOnRails, git, vagrant.

Результатом выполнения дипломного проекта является информационная система, автоматизирующая многие аспекты предметной области, что позволяет заметно оптимизировать игровой процесс.

Также в рамках дипломного проекта был создан набор тестов, обеспечивающих высокую надёжность системы. Для автоматизированных тестов была проведена оценка покрытия кода системы, которая составила 95.81\%, что является достаточно высоким показателем.

Так как время и ресурсы на разработку проекта в рамках дипломного проектирования ограничены, не все функции удалось реализовать. Возможными направлениями развития проекта могут быть:
\begin{itemize}
\item улучшение дизайна страниц;
\item добавление сложных пользовательских функций;
\item улучшение реализованных функций в соответствии с требованиями пользователей;
\item создание мобильной версии;
\item оптимизация и реорганизация кода проекта.
\end{itemize}


\pagebreak
