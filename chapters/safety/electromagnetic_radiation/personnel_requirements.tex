\subsubsection{Требования к персоналу, эксплуатирующему средства вычислительной техники}

К самостоятельной эксплуатации электроаппаратуры допускается только специально обученный персонал не моложе 18 лет, пригодный по состоянию здоровья и квалификации к выполнению указанных работ.

Перед допуском к работе персонал должен пройти вводный и первичный инструктаж по технике безопасности с показом безопасных и рациональных примеров работы. Затем не реже одного раза в 6 месяцев проводится повторный инструктаж, возможно, с группой сотрудников одинаковой профессии в составе не более 20 человек. Внеплановый инструктаж проводится при изменении правил по охране труда, при обнаружении нарушений персоналом инструкции по технике безопасности, изменении характера работы персонала.

В помещениях, в которых постоянно эксплуатируется электрооборудование должны быть вывешены в доступном для персонала месте инструкции по технике безопасности, в которых также должны быть определены действия персонала в случае возникновения аварий, пожаров, электротравм.

Руководители структурных подразделений несут ответственность за организацию правильной и безопасной эксплуатации средств вычислительной техники и периферийного оборудования, эффективность их использования; осуществляют контроль за выполнением персоналом требований настоящей инструкции по технике безопасности.

\subsubsection{Расчет освещенности}

Профессиональным заболеванием операторов и программистов является ухудшение зрения.
Так как параметры помещения, в котором ведется работа, и используемой техники (персональной ПЭВМ) удовлетворяют санитарным нормам, то особое внимание следует уделить освещенности рабочего места.

Обычно искусственное освещение выполняется посредством электрических источников света двух видов: ламп накаливания и люминесцентных ламп. Будем использовать люминесцентные лампы, которые по сравнению с лампами накаливания имеют ряд существенных преимуществ:
\begin{itemize}
\item по спектральному составу света они близки к дневному, естественному свету;
\item обладают более высоким КПД (в 1,5-2 раза выше, чем КПД ламп накаливания);
\item обладают повышенной светоотдачей (в 3-4 раза выше, чем у ламп накаливания);
\item более длительный срок службы.
\end{itemize}

Расчет освещенности проводится с путем расчета коэффициента использования с использованием метода светового потока. Он позволяет учесть прямую и отраженную составляющую светового потока от потолка, стен и рабочих поверхностей.

Имеются следующие исходные данные:
\begin{itemize}
  \item площадь помещения --- 5x3 м;
  \item высота подвеса светильников $h_\textup{св}$= 2,8 м;
  \item источник освещения --- лампа люминесцентная (ЛБА), яркость фона --- светлая;
  \item яркость объекта --- средняя;
  \item система освещения --- общая;
  \item коэффициент отражения побеленного потолка: $p_\textup{п}$= 0,6;
  \item коэффициент отражения стен, обклеенных обоями: $p_c$ = 0,4;
  \item коэффициент отражения расчетной поверхности: $p_p$ = 0,3.
\end{itemize}

По таблице <<Нормы освещенности при искусственном освещении и коэффициент естественного освещения (для 3 пояса светового климата РФ) при естественном и совмещенном освещении>> (СНиП 23-05-95), исходя из характеристик зрительной работы определяем разряд и подразряд зрительной работы как IV-В.
Данному разряду соответствует норма искусственного освещения при системе комбинированного освещения 300 лк.

Норма рабочего искусственного освещения составляет $E_\textup{он}$ = 400 лк. Коэффициент запаса равен $K_3$ = 1,5, высота подвеса светильников $h_\textup{св}$ = 2,8 м.

Определяем индекс помещения:

\formulka{ i = {{a \cdot b} \over {(a + b) \cdot h_\textup{св}}} = {{5 \cdot 3} \over {(5 + 3) \cdot 2.8}} = {\approx 0,64} }

Тип лампы --- ЛБА, люминесцентная белого света, амальгамная.
Интерполированием находим коэффициент использования: $\eta$ = 0,22.

Определяем суммарный световой поток.
Коэффициент минимальной освещенности, зависящий от размещения и светораспределения светильников, создающих некоторую неравномерность распределения светового потока по расчетной плоскости, принимаем равным $Z$ = 1,1.

\formulka{ \Phi_c = {{E \cdot S \cdot Z \cdot K_3} \over {n \cdot \eta}} = {{400 \cdot 15 \cdot 1,1 \cdot 1,5} \over {0,22}} = 45000 (\textup{лм}) }

Выбираем лампу ЛБ-80 со световым потоком 5200 лм, срок продолжительности горения 12000 час, мощность 80 Вт.

Количество необходимых ламп рассчитывается как отношение суммарного светового потока к световому потоку одной лампы:

\formulka{ n = {\Phi_c \over \Phi_{\text{л}}} = {45000 \over 5200} \approx 9 }

Таким образом, на площадь 5 x 3 м для работы за дисплеем при общем освещении должны использоваться 9 светильников по 1 лампе ЛБА-80.
