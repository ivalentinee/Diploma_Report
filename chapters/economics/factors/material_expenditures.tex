\subsubsection{Материальные затраты}

\paragraph{Затраты на сырье и материалы}
Затратами на сырьё и материалы являются затраты на специализированные устройства с дисплеями (три планшета, три смартфона) и расходные материалы, включающие в себя бумагу для печати, канцтовары и одну заправку картриджа лазерного ЧБ принтера, необходимую для печати.

\begin{longtable}[h]{| p{0.2\textwidth} | p{0.2\textwidth} | p{0.2\textwidth} | p{0.2\textwidth} | p{0.2\textwidth} |}
\hline
 Параметр                      &  Модель               &  Цена, руб  &  Количество единиц, шт.  &  Сумма, руб  \\
\hline
 Серверная платформа OS Linux  &  ASUS RS704D-E6/P     &  90 092     &                       1  &  90 092      \\
\hline
 Планшет с ОС Android          &  Google NEXUS 7 16gb  &  10 288     &                       1  &  10 288      \\
\hline
 Ноутбук с ОС Linux            &  Asus X501A           &  11 560     &                       1  &  11 560      \\
\hline
                               &                       &             &                   ИТОГО  &  111 940     \\
\hline
\end{longtable}



\paragraph{Затраты на энергию}
Основными материальными затратами при разработке систему информационного обеспечения и управления для ролевой игры dungeons & dragons являются затраты электроэнергии на питание персональных компьютеров, на которых производится реализация программного продукта. При мощности блока питания серверной платформы в 770 Вт затраты электроэнергии всех устройств серверной платформы будут составлять примерно \textbf{750 Вт/ч}. При использовании максимальной мощности ноутбука, затраты электроэнергии будет составлять примерно \textbf{67 Вт/ч}. При работе подключенных к сети электропитания остальных устройств (планшет) происходит расход энергии \textbf{28 Вт/ч}.

\textit{Затраты электроэнергии на ПК для реализации проекта:}\\
$Э_{ПК} = (750 + 67 + 28) \cdot 8 \cdot 21.25 \cdot 4 = 574.60 [кВт]$

Тариф на электроэнергию: 1 кВт --- 2.51 руб

\textit{Стоимость затраченной на ПК электроэнергии для реализации проекта:}\\
$Ц_{Эпк} = 574.60 \cdot 2.51 = 1442.25 [руб.]$

На рабочем месте программиста должна быть обеспечена необходимая с учётом времени года освещенность (нормы естественного, искусственного и совмещённого освещения зданий и сооружений приведены в СНиП 23-05-95 <<Естественное и искусственное освещение>>) посредством общего и местного искусственного освещения. Расход электроэнергии на искусственное освещение помещения и рабочего места программиста на период разработки программного продукта с февраля по май включительно представлены в таблице 2.

\textit{Стоимость затраченной на освещение одного места электроэнергии для реализации проекта:}\\
$Ц_{Эосв} = 78.26 \cdot 2.51 = 196.43 [руб.]$

\textbf{\textit{Общие затраты на электроэнергию:}}\\
$Ц_{ЭЛЕКТР.} = 1442.25 + 196.43 = 1638.68 [руб.]$


