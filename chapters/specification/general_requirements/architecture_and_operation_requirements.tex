\subsubsection{Требования к структуре и функционированию системы}

Создаваемая система должна быть централизованной, т.е. все данные должны располагаться в центральном хранилище. Также данная система должна быть клиент-серверной.

Доступ пользователей к системе должен осуществляться по протоколу HTTP1.1 и его расширению HTTPS с использованием специальных программ --- браузеров.

В системе предлагается выделить следующие функциональные подсистемы:
\begin{enumerate}
\item Подсистема пользователей --- предназначена для создания пользователей в системе и хранения их информации.
\item Подсистема контроля доступа --- предназначена для контроля доступа пользователей к ресурсам системы.
\item Подсистема хранения общих игровых данных --- предназначена для создания, хранения и модификации таких игровых сущностей, которые не специфичны для конкретного пользователя или игры, таких как монстры, предметы, игровые классы и пр.
\item Подсистема игр, которая должна обеспечивать управление игровым процессом.
\item Подсистема персонажей, предназначенная для создания, хранения и модификации персонажей пользователей.
\end{enumerate}

Все указанные подсистемы должны располагаться на одном сервере.
