\section{Требования к функциям, выполняемым системой}


\subsection{Аутентификация пользователей в системе}

Так как система подразумевает разграничение прав на уровне пользователей, то для работы с системой необходимо средство аутентификации.


\subsection{Авторизация пользователей}

Для обеспечения безопасности и разграничения прав необходима система авторизации, которая будет обеспечивать предоставление прав пользователю на основании владения ресурсами и принадлежности к определённой группе пользователей.


\subsection{Создание, просмотр, редактирование и удаление общеигровых данных}

В рамках подсистем хранения общеигровых данных необходимо обеспечить их создание, просмотр, редактирование и удаление.


\subsection{Управление игровым процессом}

Ключевой функцией системы является управление игровым процессом, поэтому необходимо обеспечить создание, просмотр и редактирование игр и их участников.

Также в рамках данной функции необходимо создать систему участия в играх и взаимодействия игроков.


\subsection{Создание, просмотр и редактирование записей игры}

Для обеспечения целостности игрового процесса как в пределах одной игровой сессии, так и между несколькими сессиями, следует создать средство, позволяющее в удобной форме вести записи, касающиеся действий и сюжетной линии игры.


\subsection{Создание, просмотр и редактирование сюжета игры}

В рамках подсистемы управления играми необходимо создать удобное средство для создания, просмотра и редактирования сюжета игры. Особое внимание стоит уделить алгоритмам обеспечения доступа игроков к определённым частям сюжета.

Для обеспечения этой функции также необходимо обеспечить добавление в игру общеигровых данных, таких как монстры, предметы и пр.


\subsection{Автоматизированное создание персонажей}

Создание персонажей является достаточно сложным процессом, поэтому необходимо наличие интерактивного редактора персонажей с автоматизацией расчётов.
