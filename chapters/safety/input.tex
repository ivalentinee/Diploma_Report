\subsection{Исходные данные}
\begin{longtable}[h]{|p{0.03\textwidth}|p{0.39\textwidth}|p{0.50\textwidth}|}
  \caption{Исходные данные}
  \\ \hline
	  \textbf{\No}                  &
	  \textbf{Наименование}         &
	  \textbf{Фактическое значение}
	\\ \hline
  \endfirsthead

  \multicolumn{3}{r}{Продолжение таблицы \thetable{}}
  \\ \hline
	  \textbf{\No}                  &
	  \textbf{Наименование}         &
	  \textbf{Фактическое значение}
	\\ \hline
  \endhead

    1                                                      &
    Тема дипломного проекта                                &
    Система управления игровым процессом для настольной ролевой игры <<Dungeons \& Dragons 3.5>>
  \\ \hline
    2                                     &
    Фамилия И.О. студента, учебная группа &
    Ионов В.С., ИСТд-51
  \\ \hline
    3                             &
    Вид технологического процесса &
    Разработка ПО с помощью ПЭВМ
  \\ \hline
    4                                   &
    Вид оборудования, паспортные данные &
    ПЭВМ
  \\ \hline
    5                                              &
    Напряжение, режим нейтрали электрического тока &
    220 В, 50 Гц, с заземлением
  \\ \hline
    6 &
    Характеристика производственного помещения по электробезопасности &
    Согласно ГОСТ 12.1.019-79, электрооборудование помещений относится к 1 классу защиты по поражению электрическим током: имеется рабочая изоляция, элемент для заземления и провод без зануляющей шины для подсоединения к источнику питания.
    По степени опасности относится к доступным условиям труда в соответствии с ГОСТ 12.1.030-81
  \\ \hline
    7 &
    Характеристика среды помещения &
    Допустимые показатели микроклимата помещения соответствуют ГОСТ 12.1.005-88.
    Уровень звукового давления (45 дБ) меньше максимального допустимого уровня (согласно СанПиН 2.2.2./2.4.1340-03, допустимый уровень звукового давления при работе на ВДТ (видиодисплейный терминал) и ПЭВМ не должен превышать 60 дБ).
  \\ \hline
    8 &
    Признаки отнесения объекта проектирования к опасным объектам &
    Нет
  \\ \hline
    9 &
    Категория производства по взрывопожарной опасности &
    В соответствии с ОНТП 24-86 помещение относится к категории В (помещение содержит горючие и трудногорючие жидкости, твердые горючие и трудногорючие вещества в малом количестве и материалы, способные только гореть при взаимодействии с кислородом  воздуха).
  \\ \hline
    10 &
    Характеристика взрыво-, пожароопасных зон &
    Класс пожароопасных зон помещения относится к II-2-А (зона, в которой обращаются твердые горючие вещества) согласно ПУЭ (правила устройства электроустановок).
  \\ \hline
    11 &
    Категория взрывоопасных смесей &
    Нет
  \\ \hline
    12 &
    Профессия рабочего, эксплуатирующего объект проектирования &
    Оператор ПЭВМ
  \\ \hline
    13 &
    Классы условий труда в соответствии с картой аттестации рабочего места:\newline
    по вредности;\newline
    по травмоопасности. &
    Класс 3.1 --- вредный.\newline
    Класс 2 --- допустимый (факторы среды и трудового процесса не превышают установленных норм, а возможные изменения функционального состояния организма, вызванные усталостью, утомлением, восстанавливаются во время регламентированного отдыха)
  \\ \hline
\end{longtable}
\vspace{1cm}
