\subsection{Структура программного обеспечения и функции его компонентов}

Debian GNU/Linux был выбран в качестве операционной системы при разработке программы.

В качестве языка программирования был выбран ruby и веб-фреймворк RubyOnRails.

Для контроля версий исходного кода программы используется git и git-репозиторий bitbucket.

Вебсервером для приложения выступает unicorn.

Для разработки приложения использовались браузеры chromium, opera и iceweasel.

Средством виртуализации для данного проекта является virtualbox в связке с vagrant.

Для вёрстки появнительной записки используется \LaTeX~и пакет ESKDx.

В качестве редактора текста пояснительной записки и кода программы был выбран GNU Emacs.

Функциональное моделирование осуществлялось в Dia.

Информационное моделирование осуществлялось в AllFusion ERwin Data Modeler.
