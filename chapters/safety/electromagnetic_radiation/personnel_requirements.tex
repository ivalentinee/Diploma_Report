\subsubsection{Требования к персоналу, эксплуатирующему средства вычислительной техники}

К самостоятельной эксплуатации электроаппаратуры допускается только специально обученный персонал не моложе 18 лет, пригодный по состоянию здоровья и квалификации к выполнению указанных работ.

Перед допуском к работе персонал должен пройти вводный и первичный инструктаж по технике безопасности с показом безопасных и рациональных примеров работы. Затем не реже одного раза в 6 месяцев проводится повторный инструктаж, возможно, с группой сотрудников одинаковой профессии в составе не более 20 человек. Внеплановый инструктаж проводится при изменении правил по охране труда, при обнаружении нарушений персоналом инструкции по технике безопасности, изменении характера работы персонала.

В помещениях, в которых постоянно эксплуатируется электрооборудование должны быть вывешены в доступном для персонала месте инструкции по технике безопасности, в которых также должны быть определены действия персонала в случае возникновения аварий, пожаров, электротравм.

Руководители структурных подразделений несут ответственность за организацию правильной и безопасной эксплуатации средств вычислительной техники и периферийного оборудования, эффективность их использования; осуществляют контроль за выполнением персоналом требований настоящей инструкции по технике безопасности.

\subsubsection{Расчет освещенности}

Профессиональным заболеванием операторов и программистов является ухудшение зрения.
Так как параметры помещения, в котором ведется работа, и используемой техники (персональной ПЭВМ) удовлетворяют санитарным нормам, то особое внимание следует уделить освещенности рабочего места.

Расчет освещенности проводится с путем расчета коэффициента использования с использованием метода светового потока).
Он позволяет учесть прямую и отраженную составляющую светового потока от потолка, стен и рабочих поверхностей.

Имеются следующие исходные данные:
\begin{itemize}
  \item площадь помещения --- 2,5x4 м;
  \item высота подвеса светильников $h_\textup{св}$= 2,5 м;
  \item источник освещения --- лампа люминесцентная (ЛБА), яркость фона --- светлая;
  \item яркость объекта --- средняя;
  \item система освещения --- общая;
  \item коэффициент отражения побеленного потолка: $p_\textup{п}$= 0,7;
  \item коэффициент отражения стен, обклеенных обоями: $p_c$ = 0,5;
  \item коэффициент отражения расчетной поверхности: $p_p$ = 0,3.
\end{itemize}

По таблице <<Нормы освещенности при искусственном освещении и коэффициент естественного освещения (для 3 пояса светового климата РФ) при естественном и совмещенном освещении>> (СНиП 23-05-95), исходя из характеристик зрительной работы определяем разряд и подразряд зрительной работы как IV-В.
Данному разряду соответствует норма искусственного освещения при системе комбинированного освещения 300 лк.

Норма рабочего искусственного освещения составляет $E_\textup{он}$ = 400 лк. Коэффициент запаса равен $K_3$ = 1,5, высота подвеса светильников $h_\textup{св}$ = 2,2 м.

Определяем индекс помещения:

$i = {{a \cdot b} \over {(a + b) \cdot h_\textup{св}}} = {{2,5 \cdot 4} \over {(2,5 + 4) \cdot 2.5}} = {\approx 0,6}$

Тип лампы --- ЛБА, люминесцентная белого света, амальгамная.
Интерполированием находим коэффициент использования: $\eta$ = 0,52.

Определяем расстояние между светильниками и по нему --- число светильников в помещении.
Рекомендованное отношение $\lambda = I_\textup{св} \over h_\textup{св}$ равно 0,8--1,2. Принимаем $\lambda$ = 0,8, тогда  $I_\textup{св} = 0,8 \cdot 2,5 = 2$ м.

Число светильников при размещении по углам квадрата вычисляется по формуле:

$n = {{a \cdot b} \over {I_\textup{св}^2}} = {{2,5 \cdot 4} \over {4}} \approx 3$

Определяем световой поток одной лампы.
Коэффициент минимальной освещенности, зависящий от размещения и светораспределения светильников, создающих некоторую неравномерность распределения светового потока по расчетной плоскости, принимаем равным $Z$ = 1,1.

$\Phi_o = {{E \cdot S \cdot Z \cdot K_3} \over {n \cdot \eta}} = {{400 \cdot 10 \cdot 1,1 \cdot 1,5} \over {3 \cdot 0,52}} \approx 4231 (\textup{лм})$

Выбираем лампу ЛБ-80-7 со световым потоком 5200 лм, срок продолжительности горения 12000 час, мощность 80 Вт.
Суммарная мощность осветительной установки общего освещения:

$P = P_o \cdot n = 80 \cdot 3 = 240 (\textup{Вт})$

Таким образом, на площадь 2,5 x 4 м для работы за дисплеем при общем освещении должны использоваться 3 светильника по 1 лампе ЛБА-80-7.
