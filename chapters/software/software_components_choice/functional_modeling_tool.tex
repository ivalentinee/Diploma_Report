\subsubsection{Средство функционального моделирования}

Произведем сравнение Dia и MS Visio.

\begin{longtable}[h]{| p{0.24\textwidth} | p{0.30\textwidth} | p{0.30\textwidth} |}
\caption{\label{tab:functional_modeling}Сравнение Dia и AllFusion Process Modeler} \\
  \hline
  \textbf{Параметр}  &  \textbf{Dia}  &  \textbf{MS Visio} \\
\endfirsthead
\tableContinue{3} \\
  \hline
  \textbf{Параметр}  &  \textbf{Dia}  &  \textbf{MS Visio} \\
  \hline
\endhead
  \hline
  Поддерживаемые нотации  &  Idef0, DFD, Idef3                        &  Idef0, DFD, Idef3 \\
  \hline
  Платный                 &  Нет                                      &  Да                \\
  \hline
  Форматы экспорта        &  EPS, SVG, DXF, CGM, WMF, PNG, JPEG, VDX  &  DWG, DXF, EMZ, EMF, GIF, JPG, PNG, SVG, SVGZ, TIF, HTM, HTML, BMP, DIB, WMF \\
  \hline
\end{longtable}

Из сравнительного анализа программ в таблице~\ref{tab:functional_modeling} видно, что при почти одинаковых возможностях Dia является бесплатным, ввиду чего использование Dia более выгодно.
