\subsubsection{Требования по обеспечению пожарной безопасности}

Противопожарная защита имеет целью поиска самых экономически целесообразных и технически обоснованных способов и средств предупреждения пожаров и их ликвидации с минимальным ущербом при оптимальном использовании сил и технических средств тушения.

Пожарная безопасность --- это состояние объекта, при котором исключается возможность пожара, а в случае его возникновения используются необходимые меры по устранению негативного влияния опасных факторов пожара на людей, сооружения и материальных ценностей.

Пожарная безопасность может быть обеспечена мерами пожарной профилактики и активной пожарной защиты.
Пожарная профилактика включает комплекс мероприятий, направленных на предупреждение пожара или уменьшение его последствий.

Активная пожарная защита --- меры, обеспечивающие успешную борьбу с пожарами или взрывоопасной ситуацией.

В помещениях запрещается:
\begin{itemize}
  \item зажигать огонь;
  \item включать электрооборудование, если в помещении пахнет газом;
  \item курить;
  \item сушить что-либо на отопительных приборах;
  \item закрывать вентиляционные отверстия в электроаппаратуре.
\end{itemize}

Источниками воспламенения являются:
\begin{itemize}
  \item искра при разряде статического электричества
  \item искры от электрооборудования
  \item искры от удара и трения
  \item открытое пламя
\end{itemize}

При возникновении пожароопасной ситуации или пожара персонал должен немедленно принять необходимые меры для его ликвидации, одновременно оповестить о пожаре администрацию.

По взрывопожарной и пожарной опасности помещения делятся по категориям А, Б, В, Г, Д (в зависимости от свойств применяемых веществ и материалов). Исследуемое  помещение относится к категории В (помещение содержит горючие и трудногорючие жидкости, твердые горючие и трудногорючие вещества в малом количестве и материалы, способные только гореть при взаимодействии с кислородом воздуха).

В зависимости от пределов огнестойкости строительных конструкций установлены восемь степеней огнестойкости зданий: I, II, III, IIIа, IIIб, IV, IVа, V. Учитывая высокую стоимость оборудования, а также категорию пожароопасности, здание, в котором предусмотрено размещение ПЭВМ, должно быть отнесено к I или II степени огнестойкости согласно ГОСТ 12.1.004-89. По классификации класс пожароопасных зон относится к II-2-А (зона в которой обращаются твердые горючие вещества) согласно ПУЭ.

К горючим материалам, присутствующим в исследуемом помещении относятся: строительные материалы для акустической и эстетической отделки, двери, полы, изоляции силовых, сигнальных кабелей, обмотки радиотехнических деталей, изоляции соединительных кабелей, ячеек, блоков, панелей, жидкости для очистки элементов, узлов ПЭВМ от загрязнения и другие.

В качестве средств пожаротушения предусмотрены: огнетушитель и система автоматической пожарной сигнализации.

Работа с персональным компьютером удовлетворяет ГОСТ 12.1.004-89 в вопросах пожароопасности и взрывоопасности, следовательно, специализированнх дополнительных мер по защите не требуется.
