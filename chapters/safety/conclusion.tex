\subsection{Выводы по разделу}
В данном разделе был произведен анализ основных вредных и опасных факторов исследуемого объекта.
По результатам анализа были разработаны мероприятия по обеспечению безопасных и комфортных условияй труда оператора ЭВМ.

Для проверки соответствия рабочий условий нормативным был произведен расчет освещенности.

Были разработаны мероприятия по охране окружающей среды и противостоянию возможным чрезвычайным ситуациям.

На основании выше изложенного, при условии выполнения всех мероприятий, соблюдения норм трудовой дисциплины и распорядка дня, рабочее место оператора персональной ЭВМ можно считать соответствующим классу труда 3.1.

Такие условия труда характеризуются такими отклонениями уровней вредных факторов от гигиенических нормативов, которые вызывают функциональные изменения, восстанавливающиеся, как правило, при более длительном (чем к началу следующей смены) прерывании контакта с вредными факторами и увеличивают риск повреждения здоровья.
