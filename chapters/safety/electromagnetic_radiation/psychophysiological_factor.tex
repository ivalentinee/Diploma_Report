\subsubsection{Психофизиологический фактор}

Психофизиологическое обеспечение профессиональной деятельности является важным звеном в комплексе мероприятий, проводимых инспекторами по охране труда в интересах сохранения здоровья и профессионально надежной деятельности специалистов, повышая безопасность и эффективность трудовой деятельности. 

К психофизиологическим факторам условий труда относятся: 

\begin{itemize}
  \item напряжение зрения и внимания;
  \item интеллектуальные и эмоциональные нагрузки;
  \item длительные статические нагрузки и монотонность труда.
\end{itemize}

Зрительная система человека приспособлена для восприятия картин природы, рисунков и печатных текстов, а не для работы с дисплеем. Изображение на дисплее состоит из светящихся и мерцающих дискретных точек, что принципиально отличается от привычных глазу объектов наблюдения.

При работе на компьютере часами у глаз не бывает необходимых фаз расслабления, глазные мышцы напрягаются, их работоспособность снижается. Большую нагрузку орган зрения испытывает при вводе информации, так как пользователь вынужден часто переводить взгляд с экрана на текст и клавиатуру, находящиеся на разном расстоянии и по-разному освещенные. 

Зрительное утомление выражается в:

\begin{itemize}
  \item двоении;
  \item затуманивании зрения;
  \item кажущемся изменении окраски предметов;
  \item трудности при переносе взгляда с ближних на дальние и с дальних на ближние предметы;
  \item неприятных ощущениях в области глаз (чувства жжения, <<песка>>), покраснении век, боли при движении глаз.
\end{itemize}

Системный администратор зачастую занят монотонным трудом при набирании больших количеств кода. К сожалению, даже при всех современных методиках разработки программного обеспечения встречаются участки работы, где без этого не обойтись. 

Для уменьшения утомления, связанного с влиянием этого фактора необходимо применять оптимальные режимы труда и отдыха в течение рабочего дня. Одна из распространенных рекомендаций --- при работе за компьютером необходимо делать 15-минутные перерывы через каждые два часа, а при интенсивной работе --- через каждый час. Также рекомендуется периодически выполнять несколько физических упражнений по своему вкусу.

Рабочая поза оказывает значительное влияние на эффективность работы человека. Рабочее место --- зона, оснащенная необходимыми техническими средствами, в которой совершается трудовая деятельность исполнителя или группы исполнителей, совместно выполняющих одну работу.

Под техническими средствами понимается основное и вспомогательное оборудование, устройства техники безопасности, санитарно гигиенические, культурно-бытовые устройства, необходимые для наиболее экономичного или наиболее производственного выполнения определенных технологических операций.

Под организацией рабочего места понимается система мероприятий по оснащения рабочего места средствами и предметами труда и их размещению в определенном порядке.

Как правило, оператор ПЭВМ не имеет возможности выбирать помещение или делать кардинальные перестановки оборудования для обеспечения наиболее оптимальных условий работы.
Поэтому рекомендации данной главы следует рассматривать как исходный материал или руководство к действию для организаторов работ с использованием ПЭВМ.
Они могут быть использованы операторами, заинтересованными в усовершенствовании своих рабочих мест.
Все приведенные требования включены в нормативный документ Госкомсанэпиднадзора <<Гигиенические требования к ПЭВМ и организации работы. Санитарные правила и нормы>>, вступивший в силу в июне 2003 г. (СанПин 2.2.2/2.4.1340-03).

В производственных помещениях, в которых работа с использованием ПЭВМ является вспомогательной, температура, относительная влажность и скорость движения воздуха на рабочих местах должны соответствовать действующим санитарным нормам микроклимата производственных помещений. Рекомендуемая температура воздуха составляет 21 $С^o$, при относительной влажности 55\%, абсолютной влажности --- 10 ${\textup{г} \over \textup{м}^3}$ и скорости движения воздуха менее 0.1 ${\textup{м} \over \textup{с}}$

Планировка рабочего места должна удовлетворять требованиям удобства работы и экономии энергии и времени оператора, соблюдения правил личной и производственной безопасности.
\begin{itemize}
  \item При размещении рабочих мест с ПЭВМ расстояние между рабочими столами с видеомониторами (в направлении тыла поверхности одного видеомонитора и экрана другого видеомонитора), должно быть не менее 2,0 м, а расстояние между боковыми поверхностями видеомониторов --- не менее 1,2м.
  \item Рабочие места с ПЭВМ при выполнении творческой работы, требующей значительного умственного напряжения или высокой концентрации внимания, рекомендуется изолировать друг от друга перегородками высотой 1,5-2,0 м.
  \item Рабочие места с ПЭВМ в помещениях с источниками вредных производственных факторов должны размещаться в изолированных кабинах с организованным воздухообменом.
  \item Конструкция рабочего стола должна обеспечивать оптимальное размещение на рабочей поверхности используемого оборудования с учетом его количества и конструктивных особенностей, характера выполняемой работы. При этом допускается использование рабочих столов различных конструкций, отвечающих современным требованиям эргономики.
  \item Конструкция рабочего стула (кресла) должна обеспечивать поддержание рациональной рабочей позы при работе на ПЭВМ, позволять изменять позу с целью снижения статического напряжения мышц шейно-плечевой области и спины для предупреждения развития утомления. Тип рабочего стула (кресла) следует выбирать с учетом роста пользователя, характера и продолжительности работы с ПЭВМ.
  \item Рабочий стул (кресло) должен быть подъемно-поворотным, регулируемым по высоте и углам наклона сиденья и спинки, а также расстоянию спинки от переднего края сиденья, при этом регулировка каждого параметра должна быть независимой, легко осуществляемой и иметь надежную фиксацию.
  \item Поверхность сиденья, спинки и других элементов стула (кресла) должна быть полумягкой, с нескользящим, слабо электризующимся и воздухопроницаемым покрытием, обеспечивающим легкую очистку от загрязнений.
  \item Экран видеомонитора должен находиться от глаз пользователя на расстоянии 600 -- 700 мм, но не ближе 500 мм с учетом размеров алфавитно-цифровых знаков и символов.
\end{itemize}
