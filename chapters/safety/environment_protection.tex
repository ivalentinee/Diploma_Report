\subsection{Защита окружающей среды}

\subsubsection{Анализ воздействия компьютера на окружающую среду}
В жизненном цикле компьютерной техники можно выделить три этапа: производство, эксплуатация, утилизация.

Вопросы защиты окружающей среды в процессе производства компьютеров возникли давно и регламентируются сейчас, в частности стандартом ТСО-03 NUTEC, по которому контролируются выбросы токсичных веществ, условия работы и др.
Согласно ТСО-03 произведенное оборудование может быть сертифицировано лишь в том случае, если не только контролируемые параметры самого оборудования соответствуют требованиям этого стандарта, но и технология производства этого оборудования отвечает требованиям стандарта.

Воздействие компьютеров на окружающую среду при эксплуатации регламентировано рядом стандартов.
Выделяют две группы стандартов и рекомендаций: по безопасности и эргономике.

При утилизации старых компьютеров происходит их разработка на фракции: металлы, пластмассы, стекло, провода, штекеры.
Из одной тонны компьютерного лома получают до 200 кг меди, 480 кг железа и нержавеющей стали, 32 кг алюминия, 3 кг серебра, 1 кг золота и 300 г палладия.

Переработку промышленных отходов производят на специальных полигонах, создаваемых в соответствии с требованиями СНиП 2.01.28-85 и предназначенных для централизованного сбора обезвреживания и захоронения токсичных отходов промышленных предприятий, НИИ и учреждений.

\subsubsection{Влияние электромагнитных излучений компьютера на здоровье человека}

По обобщенным данным, у работающих за монитором от 2 до 6 часов в сутки функциональные нарушения центральной нервной системы происходят в среднем в 4,6 раза чаще, чем в контрольных группах, болезни сердечнососудистой системы --- в 2 раза чаще, болезни верхних дыхательных путей --- в 1,9 раза чаще, болезни опорно-двигательного аппарата--- в 3,1 раза чаще.
С увеличением продолжительности работы на компьютере соотношения здоровых и больных среди пользователей резко возрастает.

Глобальная информатизация и, как следствие, широкое применение компьютерной техники в современной жизни привели к необходимости контролировать эргономические параметры используемых компьютеров.
В большинстве развитых стран мирового сообщества существуют стандарты, регламентирующие допустимые параметры излучения компьютерной техники.

Вредоносность некоторых диапазонов излучений, генерируемых компьютерами, подтверждается медицинскими исследованиями.
Экспериментальные данные говорят о том, что длительное воздействие электромагнитных волн приводит к нарушениям деятельности центральной нервной, гормональной и сердечно-сосудистой систем, изменению биохимических показателей крови.
Субъективные ощущения человека, систематически подвергающегося облучению, проявляются в виде симптомов частой головной боли, утомляемости, ухудшения памяти, болевых ощущений в области сердца, желудка и других внутренних органов.

Основным источником неблагоприятного воздействия компьютера на здоровье пользователя являются мониторы на основе электронно-лучевой трубки.
Однако не стоит недооценивать и излучения, связанные с работой системного блока, источников бесперебойного питания и прочих устройств.
Все эти элементы формируют сложную электромагнитную обстановку на рабочем месте пользователя ЭВМ.

К основным факторам неблагоприятного воздействия работы с компьютером можно отнести следующие:
\begin{itemize}
  \item электромагнитное поле сложного спектрального состава в широком диапазоне частот (от 10 Гц до 1000 МГц);
  \item электростатический заряд на ЭЛТ монитора;
  \item ультрафиолетовое, инфракрасное и рентгеновское излучения;
  \item эргономические параметры экрана (блики, мерцание, контрастность).
\end{itemize}

На биологическую реакцию человека влияют такие параметры электромагнитных полей ЭВМ, как интенсивность и частота излучения, продолжительность облучения и модуляция сигнала, частотный спектр и периодичность действия.
Сочетание вышеперечисленных параметров может давать различные последствия для реакции облучаемого биологического объекта.
Кроме того, следует отметить и такие дополнительные факторы, характерные для пользователей ПЭВМ, как изменение аэроионного состава воздуха, увеличение нагрузки на зрение, стрессовые факторы, синдром длительной статической нагрузки и пр.

В настоящее время существует достаточно данных, указывающих на отрицательное влияние работы с компьютером на все жизненно важные системы человека.
Кроме того, биологический эффект электромагнитных полей в условиях длительного воздействия может, накапливаясь, стать причиной тяжелых заболеваний.

В качестве технических стандартов безопасности мониторов широко известны шведские ТСО-92, 95, 99, 03 и МРR-II.
Они ограничивают параметры излучения монитора, потребления электроэнергии, визуальные параметры.

В части электромагнитных полей стандарту МРR-II соответствуют российские санитарные нормы СанПиН 2.2.2006-05 <<Руководство по гигиенической оценке факторов рабочей среды и трудового процесса. Критерии и классификация условий труда>>.

\subsubsection{Мероприятия по защите окружающей среды}
Для охраны окружающей среды необходимо разработать и освоить оптимальную технологию утилизации устаревших или пришедших в негодность внутренних заменяемых компонентов компьютера (интегральных схем, плат, микроконтроллеров, механических частей компьютера, шлейфов и т. д.), а также внешних магнитных носителей.
Для этого на первом этапе утилизации необходимо сортировать и складировать в отдельные контейнеры отходы <<различной природы>> (отдельно провода, отдельно платы, отдельно различные механизмы, отдельно бумагу).
На втором этапе нужно отделять от неработающих деталей исправные части и использовать их в качестве запчастей для работающих изделий (если это возможно).
Оставшиеся --- сдавать в соответствующие профильные ремонтные или утилизирующие организации.
