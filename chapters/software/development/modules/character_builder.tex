\subsubsection{Модуль \textbf{Автоматизированное создание персонажей}}

Данный модуль предоставляет возможность создания, редактирования, отображения и удаления общеигровых данных с частичной автоматизацией. Модуль содержит 268 строк кода.

\begin{longtable}[h]{| p{0.42\textwidth} | p{0.08\textwidth} | p{0.42\textwidth} |}
\caption{\label{tab:character_builder_files}Файлы модуля \textbf{Автоматизированное создание персонажей}.} \\
  \hline
  \textbf{Имя файла}  &  \textbf{Число строк}  &  \textbf{Описание} \\
\endfirsthead
\tableContinue{3} \\
  \hline
  \textbf{Имя файла}  &  \textbf{Число строк}  &  \textbf{Описание} \\
  \hline
\endhead
  \hline
  app/controllers/api/ characters\_controller.rb  &  24  &  Контроллер доступа к данным персонажей API \\
  \hline
  app/controllers/web/ characters\_controller.rb  &  16  &  Контроллер доступа к данным персонажей \\
  \hline
  app/controllers/web/users/ characters\_controller.rb  &  4  &  Контроллер доступа к данным персонажей пользователя \\
  \hline
  app/models/character.rb  &  8  &  Модель персонажа \\
  \hline
  app/models/charlevel.rb  &  5  &  Модель уровня персонажа \\
  \hline
  app/repositories/ character\_repository.rb  &  4  &  Методы доступа модели персонажа \\
  \hline
  app/assets/javascripts/angular/ app.js.coffee  &  1  &  Инициализация angular-приложения \\
  \hline
  app/assets/javascripts/angular/ controllers/character\_builder.js.coffee  &  90  &  Редактор персонажа \\
  \hline
  app/views/web/characters/ index.html.haml  &  20  &  Страница списка персонажей \\
  \hline
  app/views/web/characters/ new.html.haml  &  3  &  Страница создания персонажа \\
  \hline
  app/views/web/characters/ edit.html.haml  &  5  &  Страница редактирования персонажа \\
  \hline
  app/views/web/characters/ show.html.haml  &  11  &  Страница отображения персонажа \\
  \hline
  app/views/web/characters/ \_form.html.haml  &  62  &  Форма редактирования персонажа \\
  \hline
  app/views/web/users/characters/ index.html.haml  &  15  &  Страница списка персонажей пользователя \\
  \hline
\end{longtable}

\begin{longtable}[h]{| p{0.04\textwidth} | p{0.41\textwidth} | p{0.46\textwidth} |}
\caption{\label{tab:character_builder_specification}Спецификация модуля \textbf{Автоматизированное создание персонажей}.} \\
  \hline
  \textbf{№}  &  \textbf{Название и тип элемента}  &  \textbf{Описание} \\
\endfirsthead
\tableContinue{3} \\
  \hline
  \textbf{№}  &  \textbf{Название и тип элемента}  &  \textbf{Описание} \\
  \hline
\endhead
  \hline
  \multicolumn{3}{|c|}{\textbf{Подпрограммы}} \\
  \hline
  1   &  buildResponse   &  Составляет json-описание персонажа для отправки на сервер \\
  \hline
  2   &  buildLevelsForResponse  &  Составляет json-описание уровней персонажа для отправки на сервер \\
  \hline
  3   &  buildModifiersForResponse  &  Составляет json-описание модификаторов персонажа для отправки на сервер \\
  \hline
  4   &  character.addLevel  &  Добавляет уровень персонажу \\
  \hline
  5   &  character.removeLevel  &  Удаляет последний уровень персонажа \\
  \hline
  6   &  \$scope.addSkillModifier  &  Добавляет модификатор умения для уровня \\
  \hline
  7   &  \$scope.removeSkill  &  Удаляет выбранный модификатор умения для уровня \\
  \hline
  8   &  \$scope.skillPointsLeft  &  Вычисляет количество очков умений, доступных для распределения \\
  \hline
  9   &  character.save  &  Отправляет json-описание персонажа на сервер для сохранения \\
  \hline
  10  &  character.save  &  Отправляет json-описание персонажа на сервер для сохранения \\
  \hline
  11  &  abilityModifier  &  Вычисляет модификатор способности персонажа \\
  \hline
\end{longtable}
