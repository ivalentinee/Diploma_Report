%%% \no вместо №
\newcommand{\No}{\textnumero}
%%% Для работы ESKDX
\usepackage{xecyr}
%%% Для работы шрифтов
\usepackage{xltxtra}
%%% Times New Roman - как основной шрифт
\setmainfont[Mapping=tex-text]{Times New Roman}
%%% Courier New - для моноширного текста
\setmonofont[Scale=MatchLowercase]{Courier New}
%%% Стандартные сочетания символов ---, --, << >> и т.п.
\defaultfontfeatures{Mapping=tex-text}
%%% Переносы в русских текстах
\usepackage{polyglossia}
\setdefaultlanguage{russian}
\newfontfamily\russianfont{Times New Roman}
%%% Перенос составных слов
\XeTeXinterchartokenstate=1
\XeTeXcharclass `\- 24
\XeTeXinterchartoks 24 0 ={\hskip 0pt plus0pt minus0pt}
\XeTeXinterchartoks 0 24 ={\nobreak}
%%% Подпись к рисункам вида «Рисунок 1»
\addto{\captionsrussian}{\renewcommand{\figurename}{Рисунок}}
%%% Перенос слов, которые не умещаются в строке
\sloppy
%%% Гостовские шрифты в рамках
\renewcommand{\ESKDfontShape}{\fontspec[BoldFont={GOST type B}]{GOST type A}}
\renewcommand{\ESKDfontShape}{\fontspec[ItalicFont={GOST type A Italic}]{GOST type A Italic}}

% смена типа листа по ЕСКД
\usepackage{eskdchngsheet}


\ESKDtitle{Пояснительная записка}
% Шифр проекта
%% ДП-УлГТУ-специальноть-номер студака-год-ПЗ
\ESKDsignature{ДП-УлГТУ-23020165-09/615-2014-ПЗ}
% Автор курсового проекта
\ESKDauthor{Ионов В.С.}
% Утверждено (для титульного листа по ЕСКД)
\ESKDtitleApprovedBy{ст. преподаватель каф. ИВК}{Кандаулов В.М.}
% Дата создания документа
\ESKDdate{2014/03/21}
% Учебная группа
\ESKDcolumnIX{ИСТд-51}
% Проверил
\ESKDcolumnXIfII{Кандаулов В.М.}
% Рецензент
\renewcommand{\ESKDcolumnXfIVname}{\cyr\CYRR\cyre\cyrc\cyre\cyrn\cyrz.}
\ESKDcolumnXIfIV{Войт Н.Н.}
% Утвердил
\ESKDcolumnXIfVI{Докторов А.Е.}
% Литера
\ESKDcolumnIVfI{У}
\ESKDcolumnIVfII{Р}

\renewcommand{\ESKDcolumnXfIname}{Разраб.}
\renewcommand{\ESKDcolumnXfIIname}{Пров.}
\renewcommand{\ESKDcolumnXfIVname}{Реценз.}
\renewcommand{\ESKDcolumnXfVname}{Н. контр.}
\renewcommand{\ESKDcolumnXfVIname}{Утв.}

\renewcommand{\ESKDcolumnXIVname}{Изм.}
\renewcommand{\ESKDcolumnXVname}{Лист}
\renewcommand{\ESKDcolumnXVIname}{№ Докум.}
\renewcommand{\ESKDcolumnXVIIname}{Подп.}
\renewcommand{\ESKDcolumnXVIIIname}{Дата}

\renewcommand{\ESKDcolumnIVname}{Лит.}
\renewcommand{\ESKDcolumnVIIname}{Лист}
\renewcommand{\ESKDcolumnVIIIname}{Листов}

\renewcommand{\ESKDcolumnXfIIIname}{Т. контр.}


\ESKDsectSkip{section}{30pt}{12pt}
\ESKDsectSkip{subsection}{24pt}{12pt}
\ESKDsectSkip{subsubsection}{18pt}{12pt}
\ESKDsectStyle{subsubsection}{\mdseries}
