\subsection{Защита в чрезвычайных ситуациях}

\subsubsection{Причины возможных чрезвычайных ситуаций}
Чрезвычайная ситуация --- внешне неожиданная, внезапно возникающая обстановка, которая характеризуется резким нарушением установившегося процесса, оказывающая значительное отрицательное влияние на жизнедеятельность людей, функционирование экономики, социальную сферу и окружающую среду.

В г. Ульяновске могут возникнуть чрезвычайные ситуации производственного (техногенного) и природного характера.

Чрезвычайные ситуации производственного характера:
\begin{itemize}
  \item транспортная авария (катастрофа);
  \item пожары, взрывы, с последующим горением;
  \item аварии с выбросом вредных веществ;
  \item обрушение сооружений;
  \item аварии на коммунальных системах жизнеобеспечения;
  \item аварии на электроэнергетических системах.
\end{itemize}

В этих ситуациях источниками опасности будут являться автотранспорт и железнодорожный транспорт, предприятия, в производстве которых применяются вредные вещества.

Источниками опасности являются также газопровод, автозаправочные станции, на которых сосредоточена большая емкость бензина и дизельного топлива, НИИАР в г. Димитровграде.

Чрезвычайные ситуации природного происхождения:
\begin{itemize}
  \item опасные метеорологические явления (бури, град, сильный ливень, мороз, метель, гололед, сильная жара и др.);
  \item инфекционная загрязненность р. Волги.
  \item повышение предельно допустимых концентраций вредных примесей в атмосфере;
  \item образование обширной зоны кислотных осадков.
\end{itemize}

Необходимо отметить, что ЧС природного происхождения почти не имеют влияния на работу комплекса ПЭВМ, а в основном на персонал.
Последнее происходит в случае плохой конструкции зданий, в которых расположены производственные помещения и офисы.

Работе самого комплекса ПЭВМ существенно могут помешать ЧС техногенного характера.
Возникновение ЧС этой группы может привести следующему:
\begin{itemize}
  \item физическое повреждение ПЭВМ, его частичное или полное разрушение в результате пожара, взрыва, обрушения сооружений;
  \item отключение электроэнергии от комплекса ПЭВМ в результате аварии на станции энергоснабжения;
  \item сбой в работе программы, частичная или полная потеря информации, повреждение поверхности магнитных носителей в результате воздействия электромагнитных импульсов большой мощности.
\end{itemize}

\subsubsection{Мероприятия по предотвращению чрезвычайных ситуаций}
Во избежание возникновения ЧС первой подгруппы персоналу необходимо соблюдать правила электро- и пожарной безопасности, а также регулярно проходить инструктаж по технике безопасности.

Для того чтобы избежать последствий ЧС второй подгруппы, рекомендуется использовать бесперебойный источник питания. Это позволит сохранить и закончить работу в нормальном режиме без потери какой-либо информации.

Для частичного избежания последствий ЧС третьей подгруппы можно использовать оптические носители информации, которые не подвержены воздействию электромагнитных импульсов.
При этом рекомендуется периодически делать копии необходимой информации при высокой вероятности возникновения ЧС, приводящих к последствиям третьей группы.

\subsubsection{Аппаратные средства защиты}

Под аппаратными средствами защиты понимаются специальные средства, непосредственно входящие в состав технического обеспечения и выполняющие функции защиты как самостоятельно, так и в комплексе с другими средствами, например с программными.
Можно выделить некоторые наиболее важные элементы аппаратной защиты:
\begin{itemize}
  \item защита от сбоев в электропитании;
  \item защита от сбоев серверов, рабочих станций и локальных компьютеров;
  \item защита от сбоев устройств для хранения информации;
  \item защита от утечек информации электромагнитных излучений.
\end{itemize}

Как примеры комбинаций вышеперечисленных мер можно привести защиту информации при работе в компьютерных сетях.

\subsubsection{Мероприятия по обеспечению пожарной безопасности и расчет средств пожаротушения}

При определении видов и количества первичных средств пожаротушения следует учитывать физико-химические и пожароопасные свойства горю чих веществ, их отношение к огнетушащим веществам, а также площадь производственных помещений.

Комплектование технологического оборудования огнетушителями осуществляется согласно требованиям технических условий на это оборудование или соответствующим правилам пожарной безопасности.

Выбор типа и расчет необходимого количества огнетушителей в защищаемом помещении или на объекте следует проводить в зависимости от их огнетушащей способности, предельной площади, а также класса пожара горючих веществ и материалов:
\begin{itemize}
  \item класс А --- пожары твердых веществ, в основном органического происхождения, горение которых сопровождается тлением (древесина, текстиль, бумага);
  \item класс В --- пожары горючих жидкостей или плавящихся твердых веществ;
  \item класс С --- пожары газов;
  \item класс D --- пожары металлов и их сплавов;
  \item класс Е --- пожары, связанные с горением электроустановок.
\end{itemize}

Выбор типа огнетушителя (передвижной или ручной) обусловлен размерами возможных очагов пожара.
При их значительных размерах необходимо использовать передвижные огнетушители.

В соответствии с НПБ 105-03 помещение относится к категории В (помещение содержит горючие и трудногорючие жидкости, твердые горючие и трудногорючие вещества в малом количестве и материалы, способные только гореть при взаимодействии с кислородом воздуха).
По нормам оснащения помещений ручными огнетушителями в помещениях категории В, при возможности пожара класса А, необходимо установить 2 пенных и водных огнетушителя вместимостью 10 л или 2 порошковых огнетушителя вместимостью 5 л.

Необходимости оборудования пожарного щита в помещении нет.

Меры противопожарной безопасности:

\begin{enumerate}
\item Желательно установить на рабочем месте пожарную сигнализацию (дымоулавливатели)
\item Для борьбы с возможными пожарами класса А возможна установка пожарного щита типа ЩП-А.
  Для помещения типа Д предельная площадь покрываемая одним пожарным щитом равна 1800 м.
  В комплектацию пожарного щита типа ЩП-А входят:
  \begin{itemize}
    \item огнетушители пенные и водные вместимостью 10 л --- 2 шт.,
    \item огнетушители порошковые вместимостью 10 л --- 1 шт.,
    \item лом --- 1шт.,
    \item багор --- 1 шт.,
    \item ведро --- 2 шт.,
    \item лопата штыковая --- 1 шт.,
    \item лопата совковая --- 1 шт.,
    \item емкость для хранения воды объемом 0,2 $\textup{м}^3$ --- 1 шт.
  \end{itemize}
\end{enumerate}
