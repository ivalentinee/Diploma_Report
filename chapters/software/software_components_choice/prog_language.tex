\subsubsection{Инструментальное средство разработки и язык программирования}

\paragraph{Язык программирования}

В качестве языка программирования был выбран ruby в связке с веб-фреймворком RubyOnRails.

\paragraph{Инструментальное средство разработки}

В качестве средств разработки наиболее подходящми для выбранного языка являются Aptana Studio, Rubymine, Emacs.

Для выбора средства необходимо провести сравнительный анализ.

\paragraph{\href{http://aptana.com/}{Aptana studio}}

Aptana Studio --- основанная на Eclipse IDE, предназначенная для web-проектов. Она поддерживает работу с ruby, RubyOnRails, git. Является бесплатной. Основными недостатками данной системы являются высокие требования к скорости процессора и объёму памяти ЭВМ, малые мозможности по настройке.

\paragraph{\href{http://www.jetbrains.com/ruby/}{RubyMine}}

RubyMine --- IDE от компании JetBrains. Предназначена для ruby и RubyOnRails. Имеет встроенный отладчик, средство тестирования и автоматизаци работы над проектом. По функциональности является лучшим кандидатом. Главный недостаток данной IDE --- высокая цена.

\paragraph{\href{http://www.gnu.org/software/emacs/}{Emacs}}

Emacs --- свободный бесплатный редактор с открытым кодом. Основное его приемущуство --- расширяемость. Для работы с ruby, RubyObRails, git, vagrant и \LaTeX~существуют бесплатные дополнения, созданные сообществом. Благодаря простоте Emacs менее требователен к ресурсам, нежели Aptana Studio или RubyMine.

Наиболее перспективным для разработки приложения является Emacs, так как он бесплатный, нетребовательный к ресурсам, имеет широкие возможности настройки и обладает всем необходимым функционалом.
