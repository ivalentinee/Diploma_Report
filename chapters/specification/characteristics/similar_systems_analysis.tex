\subsection{Анализ аналогичных разработок}


\subsubsection{\href{http://www.pathguy.com/cg35.htm}{Javascript D\&D 3.5 Character Generator}}
Данное средство предоставляет средство автоматизации для расчёта создаваемого персонажа. Предоставляется в интерактивной веб-страницы. Частично ведёт расчёт параметров и характеристик персонажа. После создания персонажа создаётся подходящий для печати лист.


\subsubsection{Dungeon\&Dragons E-Tools}
Dungeon\&Dragons E-Tools является десктопным приложением, которое позволяет автоматизированно создавать и генерировать персонажей, монстров, классы, расы и другой внутриигровой контент. Для работы приложения необходима ОС Windows.


\subsubsection{\href{http://www.wizards.com/dnd/Tool.aspx?x=dnd/4new/tool/adventuretools}{Dungeons \& Dragons Insider Adventure Tools}}
Данное средство автоматизации игрового процесса предоставляется официальным разработчиком игровой системы \dnd\ 4 редакции. Оно предоставляется всем подписчикам \href{http://www.wizards.com/dnd}{специализированного ресурса}. (добавить ссылку?)

\href{http://www.wizards.com/dnd/Tool.aspx?x=dnd/4new/tool/adventuretools}{Dungeons \& Dragons Insider Adventure Tools} предоставляет следующие возможности:
\begin{enumerate}
\item Автоматизированное создание персонажа\\
Для создания персонажа данное средство предоставляет интерактивный редактор, позволяющий частично автоматизировать расчёты при создании персонажа. Также данные редактор предоставляет возможность создания листа персонажа по его параметрам и характеристикам.
\item Генерация способностей персонажа\\
В соответствии с требованиями, которые игрок вводит в систему, данное средство способно генерировать набор характеристик.
\item Генерация имён персонажей
\item Генерация монстров
\item Просмотр правил игры
\end{enumerate}

Описанные средства имеют ряд недостатков.
\textbf{\href{http://www.pathguy.com/cg35.htm}{Javascript D\&D 3.5 Character Generator}} не смотря на возможность частично автоматизировать расчёт персонажа, для использования данного средства необходомо достаточно хорошо разбираться в правилах. Также данное средство не позволяет сохранять персонажей, в результате чего невозможно редактирование персонажа после получения готового листа.
Для работы \textbf{Dungeon\&Dragons E-Tools} необходима ОС Windows, что ограничивает применение этого средства для игры.
Для использования \textbf{\href{http://www.wizards.com/dnd/Tool.aspx?x=dnd/4new/tool/adventuretools}{Dungeons \& Dragons Insider Adventure Tools}} необходимо специальное дополнение для браузера от Microsoft, что нарушает принцип кроссбраузерности и ограничивает его повсеместное использование.
В \textbf{\href{http://www.wizards.com/dnd/Tool.aspx?x=dnd/4new/tool/adventuretools}{Dungeons \& Dragons Insider Adventure Tools}} и \textbf{Dungeon\&Dragons E-Tools} отсутствует какая-либо интегрированность компонент, в результате чего построение целостной системы персонажей и игр невозможно. Каждый инструмент обладает ограниченным функционалом, предоставляя минимальный необходимый набор функций.

Также стоит отметить, что \href{http://www.wizards.com/dnd/Tool.aspx?x=dnd/4new/tool/adventuretools}{Dungeons \& Dragons Insider Adventure Tools} предоставляет инструменты только для 4 редакции ролевой системы.
