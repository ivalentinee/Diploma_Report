\ESKDstyle{empty}
%% \begin{titlepage}

\newgeometry{left=1.4cm, right=1.4cm, top=2cm, bottom=2cm}

\small
\begin{center}
  \uppercase{\textbf{министерство образования и науки российской федерации}}\\
  федеральное государственное бюджетное образовательное учреждение высшего профессионального образования\\
  \uppercase{\textbf{ульяновский государственный технический университет}}
\end{center}
\begin{tabular}{p{8.9cm} p{8.9cm}}
  Факультет \underline{\em{ИСТ}\hspace{6cm}} & Кафедра \uline{\em{ИВК}\hfill}
\end{tabular}
\begin{tabular}{p{\linewidth}}
  Специальность \uline{\em{информационные системы и технологии}\hfill}
\end{tabular}
\begin{tabular}{p{\linewidth-7cm} p{6.5cm}}
  &
  {\centering \uppercase{утверждаю:}\\*}
  {\raggedleft
  Зав. кафедрой \underline{\hspace{3.5cm}}\\
  <<\underline{\hspace{1cm}}>>\underline{\hspace{2.5cm}} 2014 г.\par
  }
\end{tabular}

\begin{center}
  \uppercase{задание}\\
  \textbf{по дипломному проекту студента}
\end{center}

\noindent\uline{\em{Ионов Валентин Сергеевич, гр. ИСТд-51}\hfill}\\
\small
{\centering (Ф.И.О., группа) \par}
\normalsize

\noindent 1. Тема проекта: \uline{Система управления игровым процессом для настольной ролевой игры <<Dungeons \& Dragons 3.5>>\hfill}\\
\noindent утверждена приказом по университету \No~\underline{\hspace{1.5cm}} от <<\underline{\hspace{1cm}}>>\underline{\hspace{3.5cm}} 20\underline{\hspace{0.8cm}} г.\\
\noindent 2. Срок сдачи студентом законченного проекта: \hspace{0.8cm} <<\underline{\hspace{1cm}}>>\underline{\hspace{3.5cm}} 20\underline{\hspace{0.8cm}} г.\\
\noindent 3. Исходные данные к проекту:
\uline{создать информационную систему, автоматизирующую игровые процессы и предоставляющую средства управления игровым процессом в рамках web-интерфейса в составе функциональных модулей: аутентификации, общеигровых данных, управления игровым процессом, создания персонажей.\hfill}\\
\noindent 4. Содержание пояснительной записки (перечень подлежащих разработке вопросов):
\uline{техническое задание; моделирование исходной информационной системы; информационное, математическое, программное, техническое обеспечение системы; тестирование системы.\hfill}\\
\noindent 5. Перечень графического материала (с точным указанием обязательных чертежей):
\uline{схема взаимодействия модулей системы (чертёж); диаграмма исходной информационной системы (чертёж); логическая модель данных (чертёж); диаграмма алгоритма вычисления очков умений (чертёж).\hfill}\\

\noindent 6. Консультанты по проекту, с указанием относящихся к ним разделов проекта:
\uline{Рыбкина М.В. – Экономический раздел, Куклев В.А. – Безопасность и экологичность.\hfill}

\restoregeometry

%% \end{titlepage}
