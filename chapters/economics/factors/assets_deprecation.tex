\subsubsection{Амортизация основных производственных средств}
Амортизационные отчисления на серверную платформу в процессе реализации проекта осуществляется пропорционально общей стоимости и времени срока службы до ее списания относительно продолжительности проекта.


\begin{longtable}[h]{| p{0.3\textwidth} | p{0.4\textwidth} | p{0.2\textwidth} |}
\caption{\label{tab:pc_cost}Стоимость персонального компьютера.} \\
  \hline
   Параметр                      &  Модель               &  Цена, руб      \\
\endfirsthead
\tableContinue{3}
  \\ \hline
\endhead
  \hline
   Ноутбук с ОС Linux            &  Asus X501A           & 11 560          \\
  \hline
  \multicolumn{2}{|r|}{\textbf{ИТОГО}}                   & \textbf{11 560} \\
  \hline
\end{longtable}


\begin{longtable}[h]{| p{0.15\textwidth} | p{0.2\textwidth} | p{0.1\textwidth} | p{0.2\textwidth} | p{0.2\textwidth} |}
\caption{\label{tab:workplace_cost}Стоимость устройств на одно рабочее место.} \\
  \hline
   Параметр                      &  Модель               &  Цена, руб &  Число пользователей &  \textbf{ИТОГО, руб.} \\
\endfirsthead
\tableContinue{5}
  \\ \hline
\endhead
  \hline
   Планшет с ОС Android          &  Google NEXUS 7 16gb &  10 288     &  10                  &  \textbf{1 028.80}    \\
  \hline
   Ноутбук с ОС Linux            &  ASUS RS704D-E6/P    &  90 092     &  10                  &  \textbf{9 009.20}    \\
  \hline
\end{longtable}


Срок амортизации ПК – 5 лет. Произведём расчёт амортизации ПК на 4 месяца (длительность реализации проекта):

\formulka{ \text{Ц}_{\text{ПК А}} = {{(18 018 + 1 028.80 + 1 156) \cdot 4} \over {5 \cdot 12}} = 1 346.65 \fUnitT{руб.} }

Амортизация зданий и сооружений составляет 100 лет. Стоимость 1 м 2 составляет 41 750 руб. Произведём расчёт амортизационных отчислений на здания и сооружения, исходя из расчёта 10 \sqMeter площади на одно рабочее место и 5 \sqMeter общей площади здания на одного человека (коридоры, туалеты, лестничные площадки).

\formulka{ \text{Ц}_{\text{ЗД А}} = 41 750 \cdot 15 \cdot 4 / (100 \cdot 12) = 2 087.50 \fUnitT{руб.} }

\textbf{\textit{Общие затраты на амортизацию:}}

\formulka{ \text{Ц}_{\text{АМОРТ.}} = 1 346.65 + 2 087.50 = 3 434.15 \fUnitT{руб.} }
