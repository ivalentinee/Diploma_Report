\subsubsection{Электрический ток}

Действие электрического тока на органы человека может быть тепловым (ожог), механическим (разрыв тканей), химическим (электролиз) и биологическим (сокращение мышц, паралич дыхания и сердца).

При воздействии электрического тока на органы человека могут быть два вида поражения: электрические удары и электрические травмы.

Электрические удары --- это возбуждение живых тканей организма протекающим через него электрическим током, проявляющееся в непроизвольных судорожных сокращениях различных мышц тела.
Они разделяются условно на четыре группы:
\begin{enumerate}
  \item{судорожное сокращение мышц без потери сознания;}
  \item{судорожное сокращение мышц с потерей сознания;}
  \item{потеря сознания и нарушение сердечной деятельности или дыхания;}
  \item{клиническая смерть, т.е. отсутствие дыхания и кровообращения.}
\end{enumerate}

По степени опасности поражения людей током рабочее пространство, оборудование дисплеями, относится к классу помещений без повышенной опасности согласно ПУЭ.

Основное питание в помещении должно осуществляется от трехфазной цепи током с частотой 50 Гц и напряжением 220 В.
Сеть трёхфазная с заземленной нейтралью ГОСТ 12.1.030-81.

В соответствии с ГОСТ 12.1.030-81, в помещениях с электрооборудованием должна быть проложена шина защитного заземления (заземляющий проводник сечением не менее 120 мм), который должен металлически соединяться с заземляющей нейтралью электроустановок, от которой осуществляется электропитание оборудования.
Сопротивление заземляющего устройства, с которым соединяется нейтраль, должно быть не более 4 Ом.

Шина защитного заземления должна быть доступна для осмотра.

В соответствии с ГОСТ 12.1.030-81 подводка питания к дисплеям и устройствам должна осуществляться под съемным полом или в каналах.
Согласно ГОСТ 12.1.030-81, электрооборудование помещения относится к 1 классу защиты от поражения электрическим током, т.е. имеется рабочая изоляция, элемент для заземления и провод без зануляющей шины для подсоединения к источнику питания.

Для защиты персонала от попадания под опасное напряжение при неисправной изоляции необходимо предусмотреть защитное заземление, выполняемое в соответствии с ГОСТ 12.1.03-81.

Предупреждение коротких замыканий в электрической сети обеспечивается правильным выбором проводов (выбор сечения, токоведущих шин, марки проводов, видов изоляции); профилактические осмотры, ремонты.
Для быстрого отключения при коротком замыкании в комнате используются плавкие предохранители.

В случае аварийных ситуаций предусмотрено защитное отключение ПЭВМ за счет превращения тока замыкания на корпус в ток однофазного короткого замыкания с последующим срабатыванием защиты.

В рассматриваемом помещении установка всего электрооборудования выполнена в соответствии с нормами электробезопасности.
Питание осуществляется от трехфазной цепи с заземленной нетралью током с частотой 50 Гц и напряжением 220 В.

Электроснабжение всех потребителей осуществляется через главный распределительный щит.
Разводка электропитания по потребителям энергии осуществляется с помощью установленных в помещениях распределительных щитов и розеток.

Для отключения питания электрической сети в помещениях предусматриваются рубильники.

То есть, такой фактор как электробезопасность рассмотренной работы, а именно работа с персональным компьютером согласно ГОСТ 12.1.030-81 можно отнести по степени опасности к допустимым условиям труда.
В общем электробезопасность удовлетворяет ГОСТ 12.1.030-81 и дополнительных мер по защите не требуется.
