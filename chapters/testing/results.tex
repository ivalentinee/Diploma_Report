\subsection{Результаты тестирования}

Для автоматизированных тестов результаты предоставляются в виде отчёта об успешных и проваленных тестах. Отчёт выводится в виде текста в стандартный поток вывода. Успешные тесты отмечаются зелёным текстом, проваленные --- красным. Для проваленных тестов выводится описание теста и пояснение о причинах провала.

Также для тестов вычисляется покрытие. В качестве критерия покрытия используется покрытие операторов. Отчёт о покрытии выводится в виде html-страницы. На данной html-странице указывается информация об общем покрытии, покрытии каждого блока модулей (таких, как <<контроллеры>>, <<модели>> и др.), покрытие каждого файла а также покрытие отдельных строк кода. Отчёт составляется после проведения полного автоматизированного тестирования системы и записывается в файл coverage/index.html.

Тестовое покрытие кода системы составляет 95.81\%. Покрытие кода системы показано на рисунке~\ref{img:testing:results}.

\portraitImg[H]{0.78}
            {images/testing/coverage}
            {Тестовое покрытие кода системы}
            {testing:results}

