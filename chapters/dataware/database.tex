\subsection{Выбор средств управления данными}

В рамках проектирования системы для выбора СУБД были выбраны основные критерии:
\begin{itemize}
\item СУБД должна быть реляционной;
\item должна работать на ОС Linux;
\item должна использовать язык SQL для совершения операций;
\item должна поддерживать все основные типы данных.
\end{itemize}

На основе указанных критериев были выбраны четыре СУБД:
\begin{enumerate}
\item PostgreSQL.
\item MySQL.
\item SQLite.
\item Oracle Database.
\end{enumerate}

Для дальнейшего выбора был проведён анализ по дополнительным критериям.

\begin{longtable}[h]{| m{0.32\textwidth} | m{0.14\textwidth} | m{0.115\textwidth} | m{0.115\textwidth} | m{0.11\textwidth} |}
\caption{\label{tab:db_compare}Сравнение СУБД.} \\
  \hline
  \textbf{Критерий}  &  \textbf{PostgreSQL}  &  \textbf{MySQL}  &  \textbf{Oracle Database}  &  \textbf{SQLite} \\
\endfirsthead
\tableContinue{4} \\
  \hline
  \textbf{Критерий}  &  \textbf{PostgreSQL}  &  \textbf{MySQL}  &  \textbf{Oracle Database}  &  \textbf{SQLite} \\
  \hline
\endhead
  \hline
  Бесплатная              &  Да  &  Да   &  Нет  &  Да   \\
  \hline
  Поддержка массивов      &  Да  &  Нет  &  Да   &  Нет  \\
  \hline
  Подробная документация  &  Да  &  Да   &  Да   &  Да   \\
  \hline
  Высокопроизводительная  &  Да  &  Да   &  Да   &  Нет  \\
  \hline
  Поддержка репликации    &  Да  &  Да   &  Да   &  Нет  \\
  \hline
\end{longtable}

В таблице~\ref{tab:db_compare} представлен сравнительный анализ СУБД на основе дополнительных критериев. По результатам данного анализа видно, что наиболее подходящей СУБД является \textbf{PostgreSQL}.
