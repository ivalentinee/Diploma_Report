\subsubsection{Операционная система}

\paragraph{Сервер}

При выборе операционной системы для сервера были выдвинуты следующие требования:
\begin{itemize}
\item операционная система должна быть бесплатной, чтобы снизить затраты на производство и эксплуатацию;
\item код операционной системы должен быть открытым, что позволяет более эффективно решать проблемы, вызванные недочётами самой ОС;
\item операционная система должна активно развиваться.
\end{itemize}

Подходящими под данные параметры операционными системами можно назвать GNU/Linux, BSD, OpenSolaris, Plan9, Haiku, FreeDOS.

Ввиду наибольшего распространения данных систем, для сравнительного анализа были выбраны GNU/Linux и BSD.

Так как наибольший опыт имеется в разработке под ОС GNU/Linux, была выбрана именно эта ОС в виде дистрибутива Debian Linux.

\paragraph{Рабочая станция}

Для рабочих станций выбор ОС не имеет значения, так как используется виртуализация для предоставления среды разработки максимально схожей со средой сервера, на котором будет работать приложение. Для виртуализации были выбраны Virtualbox и Vagrant как самые простые, надёжные и распространённые решения.
