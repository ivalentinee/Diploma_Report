\subsection{Перечень нормативной литературы}

\begin{enumerate}
  \item{ГОСТ 12.0.003-74.ССБТ. (СТ СЭВ 790-77) Опасные и вредные производственные факторы. Классификация. М.: Изд-во стандартов, 1996.}
  \item{ГОСТ 12.1.004-91.ССБТ. Пожарная безопасность. Общие требования. М.: Изд-во стандартов, 1996.}
  \item{ГОСТ 12.1.019-79.ССБТ (СТ СЭВ 4880-84). Электробезопасность. Общие требования. М.: Изд-во стандартов, 1996.}
  \item{ГОСТ 12.1.030-81.ССБТ. Электробезопасность. Защитное заземление, зануление. М.: Изд-во стандартов, 1996.}
  \item{ГОСТ 12.1.038-82.ССБ. Электробезопасность. Предельно-допустимые значения напряжений прикосновения токов. М.: Изд-во стандартов, 1996.}
  \item{Нормы пожарной безопасности --- НПБ 105-03. Установки пожаротушения и сигнализации. Нормы и правила проектирования.}
  \item{Общесоюзные нормы технологического проектирования ОНТП 24-86., М.: МВД СССР, 1986.}
  \item{Правила пожарной безопасности в Российской Федерации --- ППБ 01-03.}
  \item{Правила устройства электроустановок. М.: Энергия, 1987.}
  \item{Руководство по гигиенической оценке факторов рабочей среды и трудовых процессов. Критерии и классификация условий труда. Р 2.2.2006-05.}
  \item{Санитарные правила и нормы. СанПиН 2.2.2./2.4.1340-03 Гигиенические требования к персональным электронно-вычислительным машинам и организации работы.}
  \item{ППБ 01-03. Противопожарные нормы.}
\end{enumerate}
