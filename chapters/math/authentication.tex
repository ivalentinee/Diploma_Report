\subsection{Аутентификация}

\subsubsection{Общая характеристика}

Алгоритм предназначен для аутентификации пользователя в системе. Входными данными для данного алгоритма являются две строки: тип провайдера и идентификатор пользователя на ресурсе провайдера.

\subsubsection{Используемые данные}

Данный алгоритм использует две таблицы: таблицу пользователей и таблицу авторизационных данных.

\subsubsection{Результаты выполнения}

Результатом выполнения является запись об аутентифицированном пользователе в системе и строка, которой пользователь может подтвердить факт аутентификации --- аутентификационный код.

\subsubsection{Логическое описание}

Логическое описание представлено на рисунке~\ref{img:math:authentication}.

%% Получить входные данные.
%% Если запись о пользователе существует в системе, то предоставить session_token
%% Иначе
%%   Занести аутентификационные данные в базу
%%   Создать запись о пользователе, соответствующую аутентификационным данным
%%   Предоставить session_token

\portraitImg[h!]{0.6}
            {images/math/authentication}
            {Диаграмма алгоритма аутентификации}
            {math:authentication}
