\subsubsection{Порядок и особенности работы}

Для работы с системой необходимо открыть страницу рабочего приложения. Затем необходимо аутентифицироваться, нажав на кнопку <<Sign in with Google>> (рис.~\ref{img:workflow:sign-in}). Далее для завершения нужно следовать инструкциям Google, которые появятся на экране.

\portraitImg[h!]{0.6}
            {images/workflow/sign-in}
            {Аутентификация}
            {workflow:sign-in}

После прохождения аутентификации откроется главная страница (рис.~\ref{img:workflow:homepage}), через которую можно получить доступ к основным функциям системы.

\portraitImg[h!]{0.6}
            {images/workflow/home-page}
            {Главная страница}
            {workflow:homepage}

Для того, чтобы перейти в панель управления пользователями, необходимо воспользоваться ссылкой <<Users>>.

Панель управления пользователями (рис.~\ref{img:workflow:users}) позволяет редактировать пользователей системы. Для отображения пользователя, нужно перейти по ссылке с его именем, для удаления и редактирования по ссылкам <<destroy>> и <<edit>> соответственно.

\portraitImg[h!]{1}
            {images/workflow/users}
            {Панель управления пользователями}
            {workflow:users}

Для того, чтобы воспользоваться редактором персонажа, нужно перейти на панель управления персонажами, нажав на ссылку <<Characters>>. Также на панель управления персонажами можно перейти со страницы пользователя.

Панель управления персонажами (рис.~\ref{img:workflow:characters}) позволяет смотреть, редактировать и удалять персонажей.

\portraitImg[h!]{1}
            {images/workflow/characters}
            {Панель управления персонажами}
            {workflow:characters}

Для создания нового персонажа нужно нажать на кнопку <<New character>>.

\portraitImg[h!]{0.8}
            {images/workflow/character-builder}
            {Редактор персонажа}
            {workflow:character-builder}

Редактор персонажа (рис.~\ref{img:workflow:character-builder}) представляет из себя интерактивную страницу, которая позволяет изменять параметры и характеристики персонажа, назначать уровни.

Редактор также позволяет распределять очки умений. Для назначения очков одному из умений нужно нажать на кнопку <<Increase skill>>, выбрать умение и определить количество очков, которые нужно потратить на умение.

Для сохранения персонажа в базе необходимо нажать кнопку <<Save>>.

Для того, чтобы перейти в панель управления играми, необходимо нажать на ссылку <<Games>> на главной странице.

\portraitImg[h!]{0.8}
            {images/workflow/games}
            {Панель управления играми}
            {workflow:games}

Панель управления играми (рис.~\ref{img:workflow:games}) позволяет смотреть, редактировать и удалять игры.

\portraitImg[h!]{0.8}
            {images/workflow/game}
            {Панель управления игры}
            {workflow:game}

Для того, чтобы перейти к управлению игрой, нужно нажать на имя игры.

На панели управления игрой (рис.~\ref{img:workflow:game}) можно пригласить игрока в игру, нажав на кнопку <<Send invitation>> и выбрать пользователя. Также на данной странице можно оставлять комментарии.

%% \portraitImg[h!]{0.7}
%%             {images/workflow/sessions}
%%             {Игровые сессии}
%%             {workflow:sessions}

Для перехода на страницу игровых сессий, нужно нажать на ссылку <<Sessions>>.

На панели сессий можно создавать, удалять и просматривать игровые сессии. Механизм работы аналогичен механизму работы с другими ресурсами системы.

На странице сессии можно оставлять записи, повествующие о ходе игры.
