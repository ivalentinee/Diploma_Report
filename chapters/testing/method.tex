\subsection{Условия и порядок тестирования}

Тестирование проектируемой информационной системы предполагает тестирование прикладного программного обеспечения.

В качестве уровня тестирования были выбраны уровени модульного и фунционального (межмодульного) тестирования, так как данные уровни позволяет наиболее точно описать поведение системы при использовании автоматизированного тестирования. Модульный уровень тестирования системы используется для проверки правильности поведения моделей. Функциональный уровень тестирования системы используется для проверки правильности контроллеров.

Для тестирования системы используется тестовый фреймворк Minitest, так как он предоставляет простой интерфейс, позволяющий описать все необходимые условия.

Все тесты используют модель чёрного ящика.

Для запуска автоматизированных тестов необходимо иметь готовую среду для запуска приложения.

Перед запуском тестов необходимо перейти в директорию проекта
\blockQuote{cd~<путь до приложения>.}

Для запуска автоматизрованных тестов можно выполнить следующие команды:
\blockQuoteCommented{bundle exec rake test:models}{тестирование модулей;}
\blockQuoteCommented{bundle exec rake test:functional}{функциональное тестирование;}
\blockQuoteCommented{bundle exec rake test}{полное тестирование.}

