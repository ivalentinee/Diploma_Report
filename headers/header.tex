% Основные модули
\usepackage{etoolbox}
\usepackage{calc}
\usepackage{cmap} % для кодировки шрифтов в pdf
\usepackage[T2A]{fontenc}
\usepackage[utf8]{inputenc}
\usepackage{cyrtimes}


\usepackage{graphicx} % для вставки картинок
\usepackage{amssymb,amsfonts,amsmath,amsthm} % математические дополнения от АМС
\usepackage[usenames,dvipsnames]{color} % названия цветов
\usepackage{makecell}
\usepackage{multirow} % улучшенное форматирование таблиц
\usepackage[table]{xcolor}
\usepackage[unicode,hidelinks]{hyperref}
\usepackage{lscape} % для альбомной ориентации страниц
\usepackage{import}
\usepackage{textcase}
\usepackage{enumitem}
\usepackage{caption}
\usepackage{ulem} % подчеркивания
\usepackage{float} % шоб картинки фигачить туда, где они должны быть
%% \usepackage[superscript]{cite} % шоб сверху бибссылки были

%%% \no вместо №
\newcommand{\No}{\textnumero}
%%% Для работы ESKDX
\usepackage{xecyr}
%%% Для работы шрифтов
\usepackage{xltxtra}
%%% Times New Roman - как основной шрифт
\setmainfont[Mapping=tex-text]{Times New Roman}
%%% Courier New - для моноширного текста
\setmonofont[Scale=MatchLowercase]{Courier New}
%%% Стандартные сочетания символов ---, --, << >> и т.п.
\defaultfontfeatures{Mapping=tex-text}
%%% Перенос составных слов
\XeTeXinterchartokenstate=1
\XeTeXcharclass `\- 24
\XeTeXinterchartoks 24 0 ={\hskip 0pt plus0pt minus0pt}
\XeTeXinterchartoks 0 24 ={\nobreak}
%%% Подпись к рисункам вида «Рисунок 1»
\addto{\captionsrussian}{\renewcommand{\figurename}{Рисунок}}
%%% Перенос слов, которые не умещаются в строке
\sloppy
%%% Гостовские шрифты в рамках
\renewcommand{\ESKDfontShape}{\fontspec[BoldFont={GOST type B}]{GOST type A}}
\renewcommand{\ESKDfontShape}{\fontspec[ItalicFont={GOST type A Italic}]{GOST type A Italic}}

% смена типа листа по ЕСКД
\usepackage{eskdchngsheet}


\ESKDtitle{Пояснительная записка}
% Шифр проекта
%% ДП-УлГТУ-специальноть-номер студака-год-ПЗ
\ESKDsignature{ДП-УлГТУ-23020165-09/615-2014-ПЗ}
% Автор курсового проекта
\ESKDauthor{Ионов В.С.}
% Утверждено (для титульного листа по ЕСКД)
\ESKDtitleApprovedBy{доцент каф. ИВК}{Кандаулов В.М.}
% Дата создания документа
\ESKDdate{2014/03/21}
% Учебная группа
\ESKDcolumnIX{ИСТд-51}
% Проверил
\ESKDcolumnXIfII{Кандаулов В.М.}
% Рецензент
\renewcommand{\ESKDcolumnXfIVname}{\cyr\CYRR\cyre\cyrc\cyre\cyrn\cyrz.}
\ESKDcolumnXIfIV{Войт Н.Н.}
% Утвердил
\ESKDcolumnXIfVI{Докторов А.Е.}
% Литера
\ESKDcolumnIVfI{У}
\ESKDcolumnIVfII{Д}

\newcommand{\ESKDtheColumnXIfIIEconomy}{Рыбкина М.В.}
\newcommand{\ESKDtheColumnXIfIISafety}{Куклев В.А.}

\renewcommand{\ESKDcolumnXfIname}{Разраб.}
\renewcommand{\ESKDcolumnXfIIname}{Пров.}
\renewcommand{\ESKDcolumnXfIVname}{Реценз.}
\renewcommand{\ESKDcolumnXfVname}{Н. контр.}
\renewcommand{\ESKDcolumnXfVIname}{Утв.}

\renewcommand{\ESKDcolumnXIVname}{Изм.}
\renewcommand{\ESKDcolumnXVname}{Лист}
\renewcommand{\ESKDcolumnXVIname}{№ Докум.}
\renewcommand{\ESKDcolumnXVIIname}{Подп.}
\renewcommand{\ESKDcolumnXVIIIname}{Дата}

\renewcommand{\ESKDcolumnIVname}{Лит.}
\renewcommand{\ESKDcolumnVIIname}{Лист}
\renewcommand{\ESKDcolumnVIIIname}{Листов}

\renewcommand{\ESKDcolumnXfIIIname}{Т. контр.}


\ESKDnewStyle{formIIEconomy}{40mm}
	\ESKDputOnStyle{formIIEconomy}{frame}{\ESKDdrawFrame}
	\ESKDputOnStyle{formIIEconomy}{stamp}{%
	  \put(\ESKDltu{\ESKDstampX},\ESKDltu{\ESKDstampY}){%
		\linethickness{\ESKDlineThick}
		\put(0,40){\line(1,0){185}}
		\put(0,30){\line(1,0){65}}
		\put(0,25){\line(1,0){185}}
		\put(135,20){\line(1,0){50}}
		\put(135,15){\line(1,0){50}}
		\linethickness{\ESKDlineThin}
		\put(0,35){\line(1,0){65}}
		\multiput(0,20)(0,-5){4}{\line(1,0){65}}
		\linethickness{\ESKDlineThick}
		\put(0,0){\line(0,1){40}}
		\put(7,25){\line(0,1){15}}
		\put(17,0){\line(0,1){40}}
		\put(40,0){\line(0,1){40}}
		\put(55,0){\line(0,1){40}}
		\put(65,0){\line(0,1){40}}
		\put(135,0){\line(0,1){25}}
		\put(140,15){\line(0,1){5}}
		\put(145,15){\line(0,1){5}}
		\put(150,15){\line(0,1){10}}
		\put(165,15){\line(0,1){10}}
		\put(67, 1){\parbox[b][23mm][c]{66mm}{\centering\ESKDfontV\ESKDtheColumnI}}
		\put(67, 26){\parbox[b][13mm][c]{106mm}{\centering\ESKDfontVII\ESKDtheColumnII}}
		\put(135, 21.3){\makebox[15mm]{\ESKDfontIII\ESKDcolumnIVname}}
		\put(135, 16.3){\makebox[5mm][c]{\ESKDfontIII\ESKDtheColumnIVfI}}
		\put(140, 16.3){\makebox[5mm][c]{\ESKDfontIII\ESKDtheColumnIVfII}}
		\put(145, 16.3){\makebox[5mm][c]{\ESKDfontIII\ESKDtheColumnIVfIII}}
		\put(150, 21.3){\makebox[15mm]{\ESKDfontIII\ESKDcolumnVIIname}}
		\put(165, 21.3){\makebox[20mm]{\ESKDfontIII\ESKDcolumnVIIIname}}
		\put(137, 1){\parbox[b][13mm][c]{46mm}{\centering\ESKDfontV\ESKDtheColumnIX}}
		\put(0.5, 21.3){\makebox[16mm][l]{\ESKDfontIII\ESKDcolumnXfIname}}
		\put(0.5, 16.3){\makebox[16mm][l]{\ESKDfontIII\ESKDcolumnXfIIname}}
		\put(0.5, 11.3){\makebox[16mm][l]{\ESKDfontIII\ESKDcolumnXfIVname}}
		\put(0.5, 6.3){\makebox[16mm][l]{\ESKDfontIII\ESKDcolumnXfVname}}
		\put(0.5, 1.3){\makebox[16mm][l]{\ESKDfontIII\ESKDcolumnXfVIname}}
		\put(17.5, 21.3){\makebox[22mm][l]{\ESKDfontIII\ESKDtheColumnXIfI}}
		\put(17.5, 16.3){\makebox[22mm][l]{\ESKDfontIII\ESKDtheColumnXIfIIEconomy}}
		\put(17.5, 11.3){\makebox[22mm][l]{\ESKDfontIII\ESKDtheColumnXIfIV}}
		\put(17.5, 6.3){\makebox[22mm][l]{\ESKDfontIII\ESKDtheColumnXIfV}}
		\put(17.5, 1.3){\makebox[22mm][l]{\ESKDfontIII\ESKDtheColumnXIfVI}}
		\put(0, 26.3){\makebox[7mm]{\ESKDfontIII\ESKDcolumnXIVname}}
		\put(7, 26.3){\makebox[10mm]{\ESKDfontIII\ESKDcolumnXVname}}
		\put(17, 26.3){\makebox[23mm]{\ESKDfontIII\ESKDcolumnXVIname}}
		\put(40, 26.3){\makebox[15mm]{\ESKDfontIII\ESKDcolumnXVIIname}}
		\put(55, 26.3){\makebox[10mm]{\ESKDfontIII\ESKDcolumnXVIIIname}}
	  }
	}

\ESKDnewStyle{formIISafety}{40mm}
	\ESKDputOnStyle{formIISafety}{frame}{\ESKDdrawFrame}
	\ESKDputOnStyle{formIISafety}{stamp}{%
	  \put(\ESKDltu{\ESKDstampX},\ESKDltu{\ESKDstampY}){%
		\linethickness{\ESKDlineThick}
		\put(0,40){\line(1,0){185}}
		\put(0,30){\line(1,0){65}}
		\put(0,25){\line(1,0){185}}
		\put(135,20){\line(1,0){50}}
		\put(135,15){\line(1,0){50}}
		\linethickness{\ESKDlineThin}
		\put(0,35){\line(1,0){65}}
		\multiput(0,20)(0,-5){4}{\line(1,0){65}}
		\linethickness{\ESKDlineThick}
		\put(0,0){\line(0,1){40}}
		\put(7,25){\line(0,1){15}}
		\put(17,0){\line(0,1){40}}
		\put(40,0){\line(0,1){40}}
		\put(55,0){\line(0,1){40}}
		\put(65,0){\line(0,1){40}}
		\put(135,0){\line(0,1){25}}
		\put(140,15){\line(0,1){5}}
		\put(145,15){\line(0,1){5}}
		\put(150,15){\line(0,1){10}}
		\put(165,15){\line(0,1){10}}
		\put(67, 1){\parbox[b][23mm][c]{66mm}{\centering\ESKDfontV\ESKDtheColumnI}}
		\put(67, 26){\parbox[b][13mm][c]{106mm}{\centering\ESKDfontVII\ESKDtheColumnII}}
		\put(135, 21.3){\makebox[15mm]{\ESKDfontIII\ESKDcolumnIVname}}
		\put(135, 16.3){\makebox[5mm][c]{\ESKDfontIII\ESKDtheColumnIVfI}}
		\put(140, 16.3){\makebox[5mm][c]{\ESKDfontIII\ESKDtheColumnIVfII}}
		\put(145, 16.3){\makebox[5mm][c]{\ESKDfontIII\ESKDtheColumnIVfIII}}
		\put(150, 21.3){\makebox[15mm]{\ESKDfontIII\ESKDcolumnVIIname}}
		\put(165, 21.3){\makebox[20mm]{\ESKDfontIII\ESKDcolumnVIIIname}}
		\put(137, 1){\parbox[b][13mm][c]{46mm}{\centering\ESKDfontV\ESKDtheColumnIX}}
		\put(0.5, 21.3){\makebox[16mm][l]{\ESKDfontIII\ESKDcolumnXfIname}}
		\put(0.5, 16.3){\makebox[16mm][l]{\ESKDfontIII\ESKDcolumnXfIIname}}
		\put(0.5, 11.3){\makebox[16mm][l]{\ESKDfontIII\ESKDcolumnXfIVname}}
		\put(0.5, 6.3){\makebox[16mm][l]{\ESKDfontIII\ESKDcolumnXfVname}}
		\put(0.5, 1.3){\makebox[16mm][l]{\ESKDfontIII\ESKDcolumnXfVIname}}
		\put(17.5, 21.3){\makebox[22mm][l]{\ESKDfontIII\ESKDtheColumnXIfI}}
		\put(17.5, 16.3){\makebox[22mm][l]{\ESKDfontIII\ESKDtheColumnXIfIISafety}}
		\put(17.5, 11.3){\makebox[22mm][l]{\ESKDfontIII\ESKDtheColumnXIfIV}}
		\put(17.5, 6.3){\makebox[22mm][l]{\ESKDfontIII\ESKDtheColumnXIfV}}
		\put(17.5, 1.3){\makebox[22mm][l]{\ESKDfontIII\ESKDtheColumnXIfVI}}
		\put(0, 26.3){\makebox[7mm]{\ESKDfontIII\ESKDcolumnXIVname}}
		\put(7, 26.3){\makebox[10mm]{\ESKDfontIII\ESKDcolumnXVname}}
		\put(17, 26.3){\makebox[23mm]{\ESKDfontIII\ESKDcolumnXVIname}}
		\put(40, 26.3){\makebox[15mm]{\ESKDfontIII\ESKDcolumnXVIIname}}
		\put(55, 26.3){\makebox[10mm]{\ESKDfontIII\ESKDcolumnXVIIIname}}
	  }
	}


\ESKDsectSkip{section}{30pt}{12pt}
\ESKDsectSkip{subsection}{24pt}{12pt}
\ESKDsectSkip{subsubsection}{18pt}{12pt}
\ESKDsectStyle{subsubsection}{\mdseries}


\newcolumntype{C}[1]{>{\centering\arraybackslash\hspace{0pt}}p{#1}}

% расстояние между элементами списка
%% \setlist[enumerate]{topsep=0pt,itemsep=-1ex,partopsep=1ex,parsep=1ex}
\setlist[itemize]{leftmargin=1.5cm,nolistsep}
\setlist[enumerate]{leftmargin=1.5cm,nolistsep}

\captionsetup[longtable]{justification=raggedright}
% чтобы работал русский для enumitem
\AddEnumerateCounter{\Asbuk}{\@Asbuk}{\CYRM}
\AddEnumerateCounter{\asbuk}{\@asbuk}{\cyrm}

\renewcommand{\labelenumi}{\arabic{enumi}.}
\renewcommand{\labelenumii}{\asbuk{enumii}.}



% Включаемый переност строки
\newtoggle{breakpage}

\newcommand{\condbreak}{
  \iftoggle{breakpage}{
    \pagebreak
  }
}

% Включает сокращения и обозначения
\newtoggle{abbreviations}

% Включает аннотацию
\newtoggle{annotation}

% Включает заключение
\newtoggle{conclusion}


\toggletrue{breakpage}
\toggletrue{abbreviations}
\toggletrue{annotation}
\toggletrue{conclusion}



\newcommand{\dnd}{D\&D}

\makeatletter
\renewcommand\paragraph{
   \@startsection{paragraph}{4}{\parindent}
      {-\baselineskip}
      {.5\baselineskip}
      {\normalfont\normalsize\bfseries}}
\makeatother


\newcommand{\formulka}[1]{
  \begin{center}
    $\displaystyle#1$
  \end{center}
}

\newcommand{\blockQuote}[1]{
  \begin{quote}
  {
    \small
    \texttt{#1}
  }
  \end{quote}
}

\newcommand{\blockQuoteCommented}[2]{
  \begin{quote}
  {
    \small
    \texttt{#1}
    \normalsize
    --- #2
  }
  \end{quote}
}

\newcommand{\fUnit}[1]{
  ~[#1]
}

\newcommand{\fUnitT}[1]{
  \fUnit{\text{#1}}
}

\newcommand*\rfrac[2]{{}^{#1}\!/_{#2}}

\newcommand{\sqMeter}{
  $\text{м}^2$
}

\newcommand{\quMeter}{
  $\text{м}^3$
}

\newcommand{\emptyString}{
  \textcolor{white}{sometext}
}

\newcommand{\portraitImg}[5][h]{
  \begin{figure}[#1]
  \center{\includegraphics[width=#2\linewidth]{#3}}
  \caption{#4}
  \label{img:#5}
  \end{figure}
}


\newcommand{\landscapeImg}[5][h]{
  \begin{landscape}
    \begin{figure}[#1]
      \center{\includegraphics[width=#2\linewidth]{#3}}
      \caption{#4}
      \label{img:#5}
    \end{figure}
  \end{landscape}
}


\newcommand{\landscapeImgCenter}[5][h]{
  \begin{landscape}
    \vspace*{\fill}
    \begin{figure}[#1]
      \center{\includegraphics[width=#2\linewidth]{#3}}
      \caption{#4}
      \label{img:#5}
    \end{figure}
    \vspace*{\fill}
  \end{landscape}
}


\newcommand{\portraitImgTwo}[6][h]{
  \begin{figure}[#1]
    \begin{minipage}[h]{\linewidth/2}
      \flushright{\includegraphics[width=#2\linewidth]{#3}}
    \end{minipage}
    \hspace{1pt}
    \begin{minipage}[h]{\linewidth/2}
      \flushleft{\includegraphics[width=#2\linewidth]{#4}}
    \end{minipage}
    \caption{#5}
    \label{img:#6}
  \end{figure}
}


\newcommand{\tableContinue}[1]{
  \multicolumn{#1}{r}{Продолжение таблицы \thetable{}}
}

