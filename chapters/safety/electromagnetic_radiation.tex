\subsection{Электромагнитное излучение}
Электромагнитным полем называется особая форма материи, посредством которой осуществляется взаимодействие между электрически заряженными частицами.

Токоведущие части действующих установок являются источником электромагнитных полей промышленной частоты.
Длительное действие электромагнитного поля на организм вызывает нарушение функционального состояния центральной нервной системы, сердечно-сосудистой системы, что приводит к быстрому утомлению, уменьшению работоспособности, болям в области сердца, изменению кровяного давления.

Параметрами степени воздействия электромагнитных полей на человека являются:
\begin{itemize}
  \item{интенсивность излучения;}
  \item{режим излучения;}
  \item{длина волны;}
  \item{размер облучаемой поверхности тела;}
  \item{особенности облучаемого организма;}
  \item{продолжительность воздействия.}
\end{itemize}

Для пользователя ПЭВМ основным источником электромагнитных полей и ионизирующих излучений является электронно-лучевая трубка дисплея.

Электромагнитные поля оказывают тепловое воздействие на организм, приводящее к структурным и функциональным изменениям в нем.

Воздействие электромагнитных полей способно вызвать изменение клеток и состава крови, замутнение хрусталика глаза, выпадение волос, ломку ногтей, ожоги, кожные заболевания и др.

Временно допустимые уровни электромагнитного потока на рабочем месте не должны превышать (для диапазона частот 5 Гц --- 2 кГц):
\begin{itemize}
  \item{плотность магнитного потока --- не более 250 нТл (по данным СанПиН 2.2.2./2.4.1340-03);}
  \item{напряженность электрического поля --- не более 25 В/м;}
  \item{радиационное излучение на расстоянии 5 см от экрана --- не более 100 мкР/ч. (по данным СанПиН 2.2.2./2.4.1340-03).}
\end{itemize}

Электромагнитные поля радиочастот делятся на высокие частоты (ВЧ), ультравысокие частоты (УВЧ), сверхвысокие частоты (СВЧ).

На пользователя, работающего за монитором с ЭЛТ-дисплеем, воздействует практически весь диапазон излучений, оказывая биологическое и тепловое воздействие.

Во всех случаях предельно допустимое значение плотности потока энергии не должно превышать 10 мкВт/$\textup{см}^2$.

Однако, в настоящее время мониторы с ЭЛТ почти повсеместно вышли из употребления, будучи заменены на более безопасные мониторы с ЖК-дисплеем.
Эти мониторы характеризуются нулевым уровнем ионизирующих излучений, однако, тем не менее, создают повышенную нагрузку на ЦНС человека за счет нагрузки на зрительную систему.

\subimport{electromagnetic_radiation/}{electric_current}

\subimport{electromagnetic_radiation/}{fire_safety}

\subimport{electromagnetic_radiation/}{psychophysiological_factor}

\subimport{electromagnetic_radiation/}{personnel_requirements}
