\subsubsection{Структура прикладного программного обеспечения}

В соответвии требованиями, указанными в разделе~\ref{sec:specification:functional_requirements}, система содержит следующие модули:
\paragraph{Аутентификация пользователей в системе}

Данный модуль предоставляет возможность аутентификации в системе. Для аутентификации используется сервис Google.

\begin{longtable}[h]{| p{0.42\textwidth} | p{0.08\textwidth} | p{0.42\textwidth} |}
\caption{\label{tab:auth_data_files}Файлы модуля \textbf{Аутентификация пользователей в системе}.} \\
  \hline
  \textbf{Имя файла}  &  \textbf{Число строк}  &  \textbf{Описание} \\
\endfirsthead
\tableContinue{3} \\
  \hline
  \textbf{Имя файла}  &  \textbf{Число строк}  &  \textbf{Описание} \\
  \hline
\endhead
  \hline
  app/controllers/web/games/ sessions\_controller.rb  &  28  &  Контроллер, отвечающий за обработку аутентификационных данных \\
  \hline
  app/models/auth\_data.rb  &  2  &  Модель, отвечающая за хранение аутентификационных данных \\
  \hline
\end{longtable}




\paragraph{Общеигровые данные}

Данный модуль предоставляет возможность создания, редактирования, отображения и удаления общеигровых данных.

\begin{longtable}[h]{| p{0.42\textwidth} | p{0.08\textwidth} | p{0.42\textwidth} |}
\caption{\label{tab:static_data_files}Файлы модуля \textbf{Общеигровые данные}.} \\
  \hline
  \textbf{Имя файла}  &  \textbf{Число строк}  &  \textbf{Описание} \\
\endfirsthead
\tableContinue{3} \\
  \hline
  \textbf{Имя файла}  &  \textbf{Число строк}  &  \textbf{Описание} \\
  \hline
\endhead
  \hline
  app/controllers/api/ properties\_controller.rb  &  12  &  Контроллен доступа к свойствам \\
  \hline
  app/controllers/web/ races\_controller.rb  &  25  &  CRUD-контроллер рас \\
  \hline
  app/controllers/web/ klasses\_controller.rb  &  27  &  CRUD-контроллер классов \\
  \hline
  app/controllers/api/ races\_controller.rb  &  8  &  Контроллен доступа к расам \\
  \hline
  app/models/klass.rb  &  3  &  Модель класса \\
  \hline
  app/models/level.rb  &  2  &  Модель уровня класса \\
  \hline
  app/models/modifier.rb  &  3  &  Модель модификатора \\
  \hline
  app/models/property.rb  &  2  &  Модель свойства \\
  \hline
  app/models/race.rb  &  3  &  Модель расы \\
  \hline
  app/repositories/ property\_repository.rb  &  5  &  Методы доступа модели свойства \\
  \hline
  app/assets/javascripts/web/ klasses.js.coffee  &  24  &  Редактор классов \\
  \hline
  app/views/api/application/ \_modifiers.json.jbuilder  &  3  &  Шаблон ответа API-контроллера списка модификаторов \\
  \hline
  app/views/api/classes/ \_levels.json.jbuilder  &  4  &  Шаблон ответа API-контроллера списка уровней класса \\
  \hline
  app/views/api/classes/ index.json.jbuilder  &  4  &  Шаблон ответа API-контроллера списка классов \\
  \hline
  app/views/api/classes/ show.json.jbuilder  &  3  &  Шаблон ответа API-контроллера класса \\
  \hline
  app/views/api/properties/ index.json.jbuilder  &  1  &  Шаблон ответа API-контроллера списка свойств \\
  \hline
  app/views/api/properties/ show.json.jbuilder  &  1  &  Шаблон ответа API-контроллера свойства \\
  \hline
  app/views/api/races/ index.json.jbuilder  &  1  &  Шаблон ответа API-контроллера списка рас \\
  \hline
  app/views/api/races/ show.json.jbuilder  &  3  &  Шаблон ответа API-контроллера расы \\
  \hline
  app/views/web/application/ \_modifier\_form.html.haml  &  7  &  Форма добавления модификатора \\
  \hline
  app/views/web/application/ \_property\_skill\_fields.html.haml  &  10  &  Форма добавления уровня навыка \\
  \hline
  app/views/web/application/ \_property\_ability\_fields.html.haml  &  10  &  Форма добавления уровня способности \\
  \hline
  app/views/web/klasses/ index.html.haml  &  13  &  Страница списка классов \\
  \hline
  app/views/web/klasses/ new.html.haml  &  4  &  Страница создания класса \\
  \hline
  app/views/web/klasses/ edit.html.haml  &  6  &  Страница редактирования класса \\
  \hline
  app/views/web/klasses/ show.html.haml  &  4  &  Страница отображения класса \\
  \hline
  app/views/web/klasses/ \_form.html.haml  &  26  &  Форма редактирования класса \\
  \hline
  app/views/web/klasses/ \_level\_fields.html.haml  &  12  &  Форма редактирования уровня класса \\
  \hline
  app/views/web/races/index.html.haml  &  13  &  Страница списка рас \\
  \hline
  app/views/web/races/new.html.haml  &  3  &  Страница создания расы \\
  \hline
  app/views/web/races/edit.html.haml  &  5  &  Страница редактирования расы \\
  \hline
  app/views/web/races/show.html.haml  &  4  &  Страница отображения расы \\
  \hline
  app/views/web/races/ \_form.html.haml  &  8  &  Форма редактирования расы \\
  \hline
\end{longtable}


\paragraph{Управление игровым процессом}

Данный модуль предоставляет возможность создания, редактирования, отображения и удаления игр и игровых сессий.

\begin{longtable}[h]{| p{0.42\textwidth} | p{0.08\textwidth} | p{0.42\textwidth} |}
\caption{\label{tab:game_management_files}Файлы модуля \textbf{Управление игровым процессом}.} \\
  \hline
  \textbf{Имя файла}  &  \textbf{Число строк}  &  \textbf{Описание} \\
\endfirsthead
\tableContinue{3} \\
  \hline
  \textbf{Имя файла}  &  \textbf{Число строк}  &  \textbf{Описание} \\
  \hline
\endhead
  \hline
  app/controllers/web/ games\_controller.rb   &   27  &  CRUD-контроллер игр \\
  \hline
  app/controllers/web/games/ sessions\_controller.rb   &   28  &  CRUD-контроллер игровых сессий \\
  \hline
  app/controllers/web/ invitations\_controller.rb   &   21  &  Контроллер приглашений в игру \\
  \hline
  app/controllers/web/games/ application\_controller.rb   &   3  &  Общие методы контроллеров игр \\
  \hline
  app/controllers/web/games/ invitations\_controller.rb   &   22  &  Контроллер доступа к приглашениям конкретной игры \\
  \hline
  app/controllers/web/games/sessions/ application\_controller.rb   &   3  &  Общие методы контроллеров игровых сессий \\
  \hline
  app/controllers/web/users/ games\_controller.rb   &   4  &  Контроллер доступа к играм конкретного пользователя \\
  \hline
  app/helpers/web/invitations\_helper.rb   &   5  &  Вспомогательные методы контроллера пришлашений в игру \\
  \hline
  app/models/game.rb   &   8  &  Модель игры \\
  \hline
  app/models/game\_character.rb   &   4  &  Модель отношения <<Персонаж --- Игра>> \\
  \hline
  app/models/invitation.rb   &   5  &  Модель приглашения в игру \\
  \hline
  app/models/session.rb   &   8  &  Модель игровой сессии \\
  \hline
  app/repositories/game\_repository.rb   &   9  &  Методы доступа модели игры \\
  \hline
  app/views/web/invitations/ index.html.haml   &   18  &  Страница списка приглашений \\
  \hline
  app/views/web/invitations/ show.html.haml   &   4  &  Страница отображения приглашения \\
  \hline
  app/views/web/invitations/ accept.html.haml   &   5  &  Форма принятия приглашения \\
  \hline
  app/views/web/games/new.html.haml   &   3  &  Страница создания игры \\
  \hline
  app/views/web/games/ index.html.haml   &   19  &  Страница списка игр \\
  \hline
  app/views/web/games/edit.html.haml   &   5  &  Страница редактирования игры \\
  \hline
  app/views/web/games/ show.html.haml   &   30  &  Страница отображения игры\\
  \hline
  app/views/web/games/ \_form.html.haml   &   10  &  Форма редактировния игры \\
  \hline
  app/views/web/games/invitations/ index.html.haml   &   12  &  Страница списка приглашений игры \\
  \hline
  app/views/web/games/invitations/ new.html.haml   &   3  &  Страница создания приглашения игры \\
  \hline
  app/views/web/games/invitations/ show.html.haml   &   3  &  Страница отображения приглашения игры \\
  \hline
  app/views/web/games/invitations/ \_form.html.haml   &   4  &  Форма редактировния приглашения игры \\
  \hline
  app/views/web/games/sessions/ index.html.haml   &   13  &  Страница списка сессий игры \\
  \hline
  app/views/web/games/sessions/ new.html.haml   &   3  &  Страница создания сессии игры \\
  \hline
  app/views/web/games/sessions/ edit.html.haml   &   5  &  Страница редактирования сессии игры \\
  \hline
  app/views/web/games/sessions/ show.html.haml   &   13  &  Страница отображения сессии игры \\
  \hline
  app/views/web/games/sessions/ \_form.html.haml   &   8  &  Форма редактировния сессии игры \\
  \hline
  app/views/web/users/games/ index.html.haml   &   12  &  Страница списка игр пользователя \\
  \hline
\end{longtable}


\paragraph{Модуль записей игры}

Данный модуль предоставляет возможность добавления заметок к играм и игровым сессиям.

\begin{longtable}[h]{| p{0.42\textwidth} | p{0.08\textwidth} | p{0.42\textwidth} |}
\caption{\label{tab:comments_files}Файлы модуля \textbf{Модуль записей игры}.} \\
  \hline
  \textbf{Имя файла}  &  \textbf{Число строк}  &  \textbf{Описание} \\
\endfirsthead
\tableContinue{3} \\
  \hline
  \textbf{Имя файла}  &  \textbf{Число строк}  &  \textbf{Описание} \\
  \hline
\endhead
  \hline
  app/controllers/web/games/ comments\_controller.rb  &  17  &  CRUD-контроллер комментариев к игре \\
  \hline
  app/controllers/web/games/sessions/ comments\_controller.rb  &  18  &  CRUD-контроллер комментариев к игровой сессии \\
  \hline
  app/models/comment.rb  &  3  &  Модель комментария \\
  \hline
  app/views/web/games/comments/ \_form.html.haml  &  3  &  Форма редактирования комментария к игре \\
  \hline
  app/views/web/games/comments/ \_comment.html.haml  &  6  &  Шаблон отображения комментария к игре \\
  \hline
  app/views/web/games/sessions/ comments/\_form.html.haml  &  3  &  Форма редактирования комментария к игровой сессии \\
  \hline
  app/views/web/games/sessions/ comments/\_comment.html.haml  &  6  &  Шаблон отображения комментария к игровой сессии \\
  \hline
\end{longtable}


\paragraph{Автоматизированное создание персонажей}

Данный модуль предоставляет возможность создания, редактирования, отображения и удаления общеигровых данных с частичной автоматизацией.

\begin{longtable}[h]{| p{0.42\textwidth} | p{0.08\textwidth} | p{0.42\textwidth} |}
\caption{\label{tab:character_builder_files}Файлы модуля \textbf{Автоматизированное создание персонажей}.} \\
  \hline
  \textbf{Имя файла}  &  \textbf{Число строк}  &  \textbf{Описание} \\
\endfirsthead
\tableContinue{3} \\
  \hline
  \textbf{Имя файла}  &  \textbf{Число строк}  &  \textbf{Описание} \\
  \hline
\endhead
  \hline
  app/controllers/api/ characters\_controller.rb  &  24  &  Контроллер доступа к данным персонажей API \\
  \hline
  app/controllers/web/ characters\_controller.rb  &  16  &  Контроллер доступа к данным персонажей \\
  \hline
  app/controllers/web/users/ characters\_controller.rb  &  4  &  Контроллер доступа к данным персонажей пользователя \\
  \hline
  app/models/character.rb  &  8  &  Модель персонажа \\
  \hline
  app/models/charlevel.rb  &  5  &  Модель уровня персонажа \\
  \hline
  app/repositories/ character\_repository.rb  &  4  &  Методы доступа модели персонажа \\
  \hline
  app/assets/javascripts/angular/ app.js.coffee  &  1  &  Инициализация angular-приложения \\
  \hline
  app/assets/javascripts/angular/ controllers/character\_builder.js.coffee  &  90  &  Редактор персонажа \\
  \hline
  app/views/web/characters/ index.html.haml  &  20  &  Страница списка персонажей \\
  \hline
  app/views/web/characters/ new.html.haml  &  3  &  Страница создания персонажа \\
  \hline
  app/views/web/characters/ edit.html.haml  &  5  &  Страница редактирования персонажа \\
  \hline
  app/views/web/characters/ show.html.haml  &  11  &  Страница отображения персонажа \\
  \hline
  app/views/web/characters/ \_form.html.haml  &  62  &  Форма редактирования персонажа \\
  \hline
  app/views/web/users/characters/ index.html.haml  &  15  &  Страница списка персонажей пользователя \\
  \hline
\end{longtable}


\paragraph{Управление учётными записями}

Модуль отвечает за создание, редактирование и удаление пользователей.

\begin{longtable}[h]{| p{0.42\textwidth} | p{0.08\textwidth} | p{0.42\textwidth} |}
\caption{\label{tab:users_files}Файлы модуля \textbf{Управление учётными записями}.} \\
  \hline
  \textbf{Имя файла}  &  \textbf{Число строк}  &  \textbf{Описание} \\
\endfirsthead
\tableContinue{3} \\
  \hline
  \textbf{Имя файла}  &  \textbf{Число строк}  &  \textbf{Описание} \\
  \hline
\endhead
  \hline
  app/controllers/web/ users\_controller.rb  &  18  &  CRUD-контроллер пользователей \\
  \hline
  app/models/user.rb  &  15  &  Модель пользователя \\
  \hline
  app/repositories/user\_repository.rb  &  5  &  Методы доступа модели пользователя \\
  \hline
  app/views/web/users/ index.html.haml  &  14  &  Страница списка пользователей \\
  \hline
  app/views/web/users/edit.html.haml  &  5  &  Страница редактирования пользователя \\
  \hline
  app/views/web/users/ show.html.haml  &  6  &  Страница отображения пользователя \\
  \hline
  app/views/web/users/ \_form.html.haml  &  10  &  Форма редактирования пользователя \\
  \hline
\end{longtable}


\paragraph{Конфигурация маршрутизатора}

Данный модуль отвечает за конфигурацию маршрутов RubyOnRails-приложения.

\begin{longtable}[h]{| p{0.42\textwidth} | p{0.08\textwidth} | p{0.42\textwidth} |}
\caption{\label{tab:routes_files}Файлы модуля \textbf{Конфигурация маршрутизатора}.} \\
  \hline
  \textbf{Имя файла}  &  \textbf{Число строк}  &  \textbf{Описание} \\
\endfirsthead
\tableContinue{3} \\
  \hline
  \textbf{Имя файла}  &  \textbf{Число строк}  &  \textbf{Описание} \\
  \hline
\endhead
  \hline
  config/routes.rb  &  29  &  Описание маршрутов приложения \\
  \hline
\end{longtable}


\paragraph{Модуль статических страниц}

Данный модуль содержит статические страницы приложения.

\begin{longtable}[h]{| p{0.42\textwidth} | p{0.08\textwidth} | p{0.42\textwidth} |}
\caption{\label{tab:static_files}Файлы модуля \textbf{Статических страниц}.} \\
  \hline
  \textbf{Имя файла}  &  \textbf{Число строк}  &  \textbf{Описание} \\
\endfirsthead
\tableContinue{3} \\
  \hline
  \textbf{Имя файла}  &  \textbf{Число строк}  &  \textbf{Описание} \\
  \hline
\endhead
  \hline
  app/views/web/application/ \_navbar.html.haml  &  7  &  Шаблон навигационной панели \\
  \hline
  app/views/web/pages/ about.en.html.haml  &  2  &  Страница <<О проекте>> --- английский вариант \\
  \hline
  app/views/web/pages/ index.en.html.haml  &  5  &  Главная страница --- английский вариант \\
  \hline
  app/views/web/pages/ about.ru.html.haml  &  2  &  Страница <<О проекте>> --- русский вариант \\
  \hline
  app/views/web/pages/ index.ru.html.haml  &  5  &  Главная страница --- русский вариант \\
  \hline
\end{longtable}


\paragraph{Файлы, обеспечивающие межмодульное взаимодействие}

Данные файлы обеспечивают передачу данных между модулями.

\begin{longtable}[h]{| p{0.42\textwidth} | p{0.08\textwidth} | p{0.42\textwidth} |}
\caption{\label{tab:other_files}Файлы, обеспечивающие межмодульное взаимодействие.} \\
  \hline
  \textbf{Имя файла}  &  \textbf{Число строк}  &  \textbf{Описание} \\
\endfirsthead
\tableContinue{3} \\
  \hline
  \textbf{Имя файла}  &  \textbf{Число строк}  &  \textbf{Описание} \\
  \hline
\endhead
  \hline
  app/controllers/api/ application\_controller.rb  &  7  &  Общие директивы API \\
  \hline
  app/helpers/web/ application\_helper.rb  &  11  &  Общие вспомогательные функции \\
  \hline
  app/controllers/web/ application\_controller.rb  &  8  &  Общие директивы приложения \\
  \hline
  app/views/layouts/web/ application.html.haml  &  12  &  Главный шаблон \\
  \hline
\end{longtable}
