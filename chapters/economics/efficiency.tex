\subsection{Расчет экономической эффективности разработки системы}

Экономическая эффективность, как правило, выступает основным интегрированным показателем успешности ведения хозяйственной деятельности для любого предприятия в любой отрасли.

В самом простом выражении экономическая эффективность производства (ЭЭП) подразумевает под собой величину соотношения того результата, который достигнут предприятием или фирмой и производственно-коммерческой деятельности и тех затрат, которые понесла данная фирма или предприятие для достижения данного результата. Количественный параметр этого соотношения называется показателем экономической эффективности и определяется как относительная результативность работы всей экономической системы для данного конкретного предприятия. Относительность параметра результативности определяется тем, что ее показатели берутся в сравнении с показателями затрат ресурсов.

Определение экономической эффективности проекта проводилось по методу расчета экономического эффекта от прибыли по формуле:

\formulka{ \text{Э}_{\text{э}} = { \text{П} \over {\text{С}_{\text{пол}}} } \times 100\% }

Экономический эффект равен:

\formulka{ \text{Э}_{\text{э}} = { 61~001.64 \over 329~008.20 } \times 100\% = 18.541069797\%}

Так как расчетный коэффициент экономической эффективности превышает нормативное значение (15 \%), следовательно, производство и внедрение данной системы считается эффективным.
