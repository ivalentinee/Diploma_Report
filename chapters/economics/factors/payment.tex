\subsubsection{Затраты на оплату труда}

Затраты на оплату труда включают:
\begin{itemize}
\item З/П программиста
\item З/П управляющего персонала
\item З/П обслуживающего персонала.
\end{itemize}

Размер заработной платы указан в таблице~\ref{tab:payment}.

Расчёт последних двух пунктов затрат по оплате труда высчитывается на одно рабочее место программиста за четыре рабочих месяца.

\begin{landscape}
\begin{longtable}[h]{| p{0.25\linewidth} | p{0.25\linewidth} | p{0.125\linewidth} | p{0.125\linewidth} | p{0.15\linewidth} |}
\caption{\label{tab:payment}Месячные затраты на З/П по реализации проекта.} \\
  \hline
  \multicolumn{2}{|c|}{\textbf{Должность}} &  \textbf{Число обслуживаемых рабочих мест} &  \textbf{З/П, руб}  &  \textbf{Затраты на З/П относительно проекта, руб} \\
\endfirsthead
\tableContinue{5}
  \\ \hline
\endhead
  \hline
   Исполнитель реализации проекта  &  Программист                     &  1     &  30~000  &  30~000         \\
  \hline
   Управляющий персонал            &  Начальник подразделения         &  20    &  30~000  &  1~500          \\ \cline{2-5}
                                   &  Зам. начальника подразделения   &  20    &  25~000  &  1~250          \\
  \hline
   Обслуживающий персонал          &  Главный системный администратор &  20    &  30~000  &  1~500          \\ \cline{2-5}
                                   &  Системный администратор         &  20    &  20~000  &  1~000          \\ \cline{2-5}
                                   &  Электрик                        &  100   &  9~000   &     90          \\ \cline{2-5}
                                   &  Инженер по технике безопасности &  500   &  10~000  &     20          \\
  \hline
  \multicolumn{4}{|r|}{\textbf{ИТОГО}}                                                    & \textbf{35~360} \\
  \hline
\end{longtable}
\end{landscape}

\textbf{\textit{Общие затраты на З/П по реализации проекта:}}
\formulka{ \text{ЦЗ}/\text{П} = 35~360.00 \cdot 4 = 141~440.00 \fUnitT{руб.} }
