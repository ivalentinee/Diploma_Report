\subsubsection{Структура прикладного программного обеспечения}

В соответвии требованиями, указанными в разделе~\ref{sec:specification:functional_requirements}, система содержит следующие модули:
\paragraph{Аутентификация пользователей в системе}

Данный модуль предоставляет возможность аутентификации в системе. Для аутентификации используется сервис Google.
%% Файлы:
app/controllers/web/games/sessions\_controller.rb : 28 lines
app/controllers/web/users/application\_controller.rb : 3 lines
app/models/auth\_data.rb : 2 lines


\paragraph{Модуль общеигровых данных}

Данный модуль предоставляет возможность создания, редактирования, отображения и удаления общеигровых данных.
%% Файлы:
app/controllers/api/properties\_controller.rb : 12 lines
app/controllers/web/races\_controller.rb : 25 lines
app/controllers/web/klasses\_controller.rb : 27 lines
app/controllers/api/races\_controller.rb : 8 lines
app/models/klass.rb : 3 lines
app/models/level.rb : 2 lines
app/models/modifier.rb : 3 lines
app/models/property.rb : 2 lines
app/models/race.rb : 3 lines
app/repositories/property\_repository.rb : 5 lines
app/assets/javascripts/web/klasses.js.coffee : 24 lines


\paragraph{Модуль управление игровым процессом}

Данный модуль предоставляет возможность создания, редактирования, отображения и удаления игр и игровых сессий.
%% Файлы:
app/controllers/web/games\_controller.rb : 27 lines
app/controllers/web/games/sessions\_controller.rb : 28 lines
app/controllers/web/invitations\_controller.rb : 21 lines
app/controllers/web/games/application\_controller.rb : 3 lines
app/controllers/web/games/invitations\_controller.rb : 22 lines
app/controllers/web/games/sessions/application\_controller.rb : 3 lines
app/controllers/web/users/games\_controller.rb : 4 lines
app/helpers/web/invitations\_helper.rb : 5 lines
app/models/game.rb : 8 lines
app/models/game\_character.rb : 4 lines
app/models/invitation.rb : 5 lines
app/models/session.rb : 8 lines
app/repositories/game\_repository.rb : 9 lines

\paragraph{Модуль записей игры}

Данный модуль предоставляет возможность добавления заметок к играм и игровым сессиям.
%% Файлы:
app/controllers/web/games/comments\_controller.rb : 17 lines
app/controllers/web/games/sessions/comments\_controller.rb : 18 lines
app/models/comment.rb : 3 lines


\paragraph{Автоматизированное создание персонажей}

Данный модуль предоставляет возможность создания, редактирования, отображения и удаления общеигровых данных с частичной автоматизацией.
%% Файлы:
app/controllers/api/characters\_controller.rb : 24 lines
app/controllers/web/characters\_controller.rb : 16 lines
app/controllers/web/users/characters\_controller.rb : 4 lines
app/models/character.rb : 8 lines
app/models/charlevel.rb : 5 lines
app/repositories/character\_repository.rb : 4 lines
app/assets/javascripts/angular/app.js.coffee : 1 line
app/assets/javascripts/angular/controllers/character\_builder.js.coffee : 90 lines


\paragraph{Управление учётными записями}
%% Файлы:
app/controllers/web/users\_controller.rb : 18 lines
app/models/user.rb : 15 lines
app/repositories/user\_repository.rb : 5 lines

\paragraph{Конфигурация маршрутизатора}
%% Файлы:
config/routes.rb : 29 lines


\paragraph{Файлы, обеспечивающие межмодульное взаимодействие}
%% Файлы:
app/controllers/api/application\_controller.rb : 7 lines
app/helpers/web/application\_helper.rb : 11 lines
app/controllers/web/application\_controller.rb : 8 lines
