В процессе работы над проектом был использован ряд государственных стандартов, интернет-ресурсов и пособий.

%% angular

%% apidock

%% wikipedia\_dnd
Статья <<Dungeons \& Dragons>> содержит краткое описание правил игры, история ее создания и развития версий. В статье так же рассматриваются ее структура игры, рассматривающая роли игроков за игровым столом. Так же статья содержит необходимую информацию о создании своего персонажа и его параметрах и процессе игры. Кроме того приведен список официальных сеттингов. Эта статья послужит хорошим источником для тех, кто хотел бы быстро узнать краткую и емкую информацию об игре. Она полезна для быстрого понимания правил и основных принципов игры, необходимых любому человеку, связанному с этой игрой.

%% DMG
Книга <<Dungeon Master's Guide. Core Rulebook II v. 3.5>> поможет изучить основы ведения игры со стороны мастера и практики ведения игровой сессии. В этой книге подробно описаны создание сцен, повествование, темп игры, информирование игроков. Мастер, читая эту книгу, обучается контролировать ход игры, проводить кампании, просчитывать необходимые ресурсы и распределять награды. В книге мастера приводятся примеры математических расчетов сцен и чудовищ, а так же шаблоны. Эта книга дает полное представление об устройстве игры изнутри. В ней можно найти все необходимые математические расчеты. Она необходима для понимания принципов создания и ведения игры. Так же она будет полезна для игроков, которые хотят более глубоко вникнуть в суть игры.

%% bootstrap

%% github

%% haml

%% html-book

%% postgresql

%% rpg\_wikia
Ресурс <<RPG Wikia>> представляет собой классическую вики - энциклопедию, в которой содержатся статьи по настольным ролевым играм на русском языке. В ней можно найти хорошо систематизированный материал по интересующим вопросам. Содержит последнюю информацию по настольным играм. Дает хорошее представление о динамике таких игр, проведении сессий. Этот ресурс позволяет быстро найти интересующую информацию, поэтому он полезен тогда, когда нужно сверять данные, хотя подходит и для детального изучения общей теории настольных ролевых игр.

%% rubydoc

%% stackoverflow

%% gost-1
<<ГОСТ Р 51904-2002. Программное обеспечение встроенных систем. Общие требования к разработке и документированию>> был подготовлен в развитие ГОСТ Р ИСО\/МЭК 12207-99 <<Информационная технология. Процессы жизненного цикла программных средств>> с целью учёта специфики разработки и документирования программного обеспечения встроенных систем реального времени и описывает процесс подготовки ПО.

%% gost-2
<<ГОСТ Р ИСО/МЭК 15408>> содержит общие критерии оценки безопасности информационных технологий. Стандарт включает в себя три главы, в них устанавливают общий подход к формированию требований безопасности, основные конструкции представления требований безопасности в интересах потребителей, разработчиков и оценщиков продуктов и систем. Также включает в себя систематизированный каталог требований доверия, определяющих меры, которые должны быть приняты на всех этапах жизненного цикла.

%% PHB
<<Руководство игрока. Книга Правил, Версия 3.5.>> является основной настольной книгой игрока. Она поможет изучить основные определения и получить базовые правила игры. Там доступно изложена вся нужная игроку информация – расы, классы, умения, навыки, описание персонажа, снаряжение. В этом источнике указаны процессы сражений и все связанные с этим вычисления. Приведены сопутствующие таблицы для расчета математических действий. В настольной книге игрока можно найти примеры на самые распространенные ситуации, возможные в игровой сессии. Данная книга будет полезна опытным и начинающим игрокам для детального изучения предметной области. Так же эта книга поможет составить полную цельную картину игры, сессий и отдельных ее аспектов.

%% clean\_code




 








