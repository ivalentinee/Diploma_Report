\section*{Введение}
\addcontentsline{toc}{section}{Введение}

Сегодня настольные ролевые игры являются достаточно популярным занятием. Существует достаточно большое количество сообществ, клубов а также интернет-ресурсов, занимающихся настольными ролевыми играми. % нет, не теми, о которых ты сейчас подумал

В настольной ролевой игре обычно участвует от двух до десяти человек, однако число участников обычно строго не регламентировано. Один из участников является <<мастером>>, то есть управляет игровым процессом. Остальные участники являются <<игроками>> и непосредственно участвуют в игре.

Ролевые настольные игры включают в себя ролевые системы. Ролевые системы в целом являются набором правил. Наиболее популярными ролевыми системами являются <<Dungeons \& Dragons>>, <<GURPS>>, <<Vampire: The Masquarade>>. Зачастую заявленные системой правила игры изменяются участниками.

Наиболе популярной ролевой системой на данный момент является Dungeons~\&~Dragons редакции 3.5.

На данный момент не существует достаточно функциональных веб-ресурсов, посвящённых игровому процессу c использованием ролевой системы Dungeons~\&~Dragons и управлению им. Наиболее остро встают проблемы автоматизации вычислений, хранения игровых данных, организации игр. Проектируемая система должна решить эти проблемы.
В рамках проектирования преследуются следующие цели:
\begin{itemize}
\item обеспечение обмена информацией между участниками;
\item создание единой базы материалов ролевой системы;
\item автоматизация вычислений.
\end{itemize}

В первом разделе пояснительной записки к дипломному проекту приводится техническое задание на разрабатываемую систему. Здесь рассматривается процесс проектирования программного обеспечения для системы автоматизированного управления ПО, формулируются основные требования к функциям разрабатываемой системы и к видам ее обеспечения, анализируется актуальность проводимой разработки.

Раздел \ref{sec:existing_system} посвящен информационному моделированию разрабатываемой системы и содержит описание ряда процессов, протекающих в ней.

Раздел \ref{sec:hardware} посвящен выбору аппаратного обеспечения для работы системы, а также для её разработки.

В разделе \ref{sec:software} описываются структура программного обеспечения, функции его компонентов, обосновывается выбор компонентов ПО, приводится описание модулей системы. Также в данном разделе описано руководство пользователя.

Тестированию системы посвящён раздел \ref{sec:testing}.

Разделы \ref{sec:economics} и \ref{sec:safety} содержат в себе аналитическую оценку экономической целесообразности проекта и оценку безопасности и экологичности проекта.

В процессе работы над проектом был использован ряд государственных стандартов, интернет-ресурсов и пособий.

<<ГОСТ Р 51904-2002. Программное обеспечение встроенных систем. Общие требования к разработке и документированию>> был подготовлен в развитие ГОСТ Р ИСО\/МЭК 12207-99 <<Информационная технология. Процессы жизненного цикла программных средств>> с целью учёта специфики разработки и документирования программного обеспечения встроенных систем реального времени и описывает процесс подготовки ПО.

<<ГОСТ Р ИСО/МЭК 15408>> содержит общие критерии оценки безопасности информационных технологий. Стандарт включает в себя три главы, в них устанавливают общий подход к формированию требований безопасности, основные конструкции представления требований безопасности в интересах потребителей, разработчиков и оценщиков продуктов и систем. Также включает в себя систематизированный каталог требований доверия, определяющих меры, которые должны быть приняты на всех этапах жизненного цикла. 

<<Руководство игрока. Книга Правил, Версия 3.5.>> является основной настольной книгой игрока. Она поможет изучить основные определения и получить базовые правила игры. Там доступно изложена вся нужная игроку информация – расы, классы, умения, навыки, описание персонажа, снаряжение. В этом источнике указаны процессы сражений и все связанные с этим вычисления. Приведено огромное количество сопутствующих таблиц, для расчета математических действий. В настольной книге игрока можно найти примеры на самые распространенные ситуации, возможные в игровой сессии. Данная книга будет полезна опытным и начинающим игрокам для детального изучения предметной области. Для игрока-мастера эта книга имеет большое значение, поскольку в ней содержится вся нужная информация о том, как нужно преподнести игровую сессию другим игрокам. Так же эта книга поможет составить полную цельную картину игры, сессий и отдельных ее аспектов.

Книга <<Dungeon Master's Guide. Core Rulebook II v. 3.5>> поможет изучить основы ведения игры со стороны мастера и практики ведения игровой сессии. В этой книге подробно описаны сложные моменты, с какими придется столкнуться мастеру. Это создание сцен, повествование, темп игры, информирование игроков. Мастер, читая эту книгу, обучается контролировать ход игры, проводить кампании, просчитывать необходимые ресурсы и распределять награды. В источнике указаны примеры описания местности, навыков, ловушек, боевых и не боевых ситуаций. В книге мастера приводятся примеры математических расчетов сцен и чудовищ, а так же шаблоны. Эта книга дает полное представление об устройстве игры изнутри. В ней можно найти все необходимые математические расчеты.Без подробного изучения данной книги, мастер, не сможет проводить игровые сессии. Она необходима для понимания принципов создания и ведения игры. Так же она будет полезна для игроков, которые хотят более глубоко вникнуть в суть игры.

Статья <<Dungeons \& Dragons>> содержит краткое описание правил игры, история ее создания и развития версий. В статье так же рассматриваются ее структура игры, рассматривающая роли игроков за игровым столом. Так же статья содержит необходимую информацию о создании своего персонажа и его параметрах и процессе игры. Кроме того приведен список официальных сеттингов. Эта статья послужит хорошим источником для тех, кто хотел бы быстро узнать краткую и емкую информацию об игре. Она полезна для быстрого понимания правил и основных принципов игры, необходимых любому человеку, связанному с этой игрой.

Ресурс <<Rolemancer>> содержит различного рода статьи по игровой механике, обзорам правил по разным редакциям. Дает подробную информацию, о действии игроков в сессии, поведении персонажей, в том числе о внутренней игровой этике, распределении ресурсов между персонажами. В статьях можно найти исчерпывающее описание игровой системы мер. В статьях указаны примеры создания персонажей, квенты, математических расчетов характеристик. Данный ресурс будет полезен игрокам, желающим подробнее изучить и вникнуть в математические аспекты игры, а так же мастерам, для предварительного просчета партии.

Ресурс <<RPG Wikia>> представляет собой классическую вики - энциклопедию, в которой содержатся статьи по настольным ролевым играм на русском языке. В ней можно найти хорошо систематизированный материал по интересующим вопросам. Содержит последнюю информацию по настольным играм. Дает хорошее представление о динамике таких игр, проведении сессий. Этот ресурс позволяет быстро найти интересующую информацию, поэтому он полезен тогда, когда нужно сверять данные, хотя подходит и для детального изучения общей теории настольных ролевых игр.

\pagebreak
