\section*{Введение}
\addcontentsline{toc}{section}{Введение}

Сегодня настольные ролевые игры являются достаточно популярным занятием. Существует достаточно большое количество сообществ, клубов а также интернет-ресурсов, занимающихся настольными ролевыми играми. % нет, не теми, о которых ты сейчас подумал

В настольной ролевой игре обычно участвует от двух до десяти человек, однако число участников обычно строго не регламентировано. Один из участников является <<мастером>>, то есть управляет игровым процессом. Остальные участники являются <<игроками>> и непосредственно участвуют в игре.

Ролевые настольные игры включают в себя ролевые системы. Ролевые системы в целом являются набором правил. Наиболее популярными ролевыми системами являются <<Dungeons~\&~Dragons>>, <<GURPS>>, <<Vampire:~The~Masquarade>>. Зачастую заявленные системой правила игры изменяются участниками.

Наиболе популярной ролевой системой на данный момент является Dungeons~\&~Dragons редакции 3.5.

На данный момент не существует достаточно функциональных веб-ресурсов, посвящённых игровому процессу c использованием ролевой системы Dungeons~\&~Dragons и управлению им. Наиболее остро встают проблемы автоматизации вычислений, хранения игровых данных, организации игр. Проектируемая система должна решить эти проблемы.
В рамках проектирования преследуются следующие цели:
\begin{itemize}
\item обеспечение обмена информацией между участниками;
\item создание единой базы материалов ролевой системы;
\item автоматизация вычислений.
\end{itemize}

\subimport{introduction/}{structure}

\subimport{introduction/}{sources_description}

\pagebreak
